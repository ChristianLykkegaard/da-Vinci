\documentclass[a4paper,11pt,danish,twoside,openany]{look/05gr551c} % Sideopstning (openany starter et nyt kapitel p vilkrlig side)openright aabner paa hoejre side mvh Per


%  Sideopstning inkl.  margin mm. 
\usepackage{anysize}	              % Set margin sizes with simple commands.
\usepackage{geometry}									%LOLOLOLOLLLLL######
\marginsize{4cm}{5cm}{2.9 cm}{2.9 cm}		% Definerer strrelse af marginer
\usepackage{setspace}										% Mulighed for dobbelt linieafstand
\usepackage[makeroom]{cancel}

\usepackage{transparent}

%Akronymliste
\usepackage[intoc]{nomencl}
\renewcommand{\nomname}{Symbol- and Acronym List}

%\makeatletter
%    \def\thenomenclature{%
%      \@ifundefined{chapter}%
%      {
%        \section*{\nomname}
%        \if@intoc\addcontentsline{toc}{section}{\nomname}\fi%
%      }%
%      {
%        \chapter*{\nomname}
%        \if@intoc\addcontentsline{toc}{chapter}{\nomname}\fi%
%      }%
%
%      \nompreamble
%      \list{}{%
%        \labelwidth\nom@tempdim
%        \leftmargin=0pt
%        \itemindent=\dimexpr\itemsep+\labelwidth\relax
%        \itemsep\nomitemsep
%        \let\makelabel\nomlabel}}
%    \makeatother

\makenomenclature
% Kr�ver at man i terminalen skriver:
%
% makeindex filename.nlo  -s nomencl.ist -o filename.nls
%
% for at printnomenclature i Rapport.tex tager input ind p� listen.
% Der oprettes et punkt p� listen et vilk�rligt sted (sammen med 
% forkortelsen er n�vnt f�rste gang) ved at inds�tte f�lgende:
%
% \nomenclature[prefix]{symbol}{description} 
% hvor prefix er optional, symbol er forkortelsen, description er det den st�r for

% This prints the Nomenclature in two columns % % % % % % % % % % % % % % % %
\usepackage{multicol}
\makeatletter
\@ifundefined{chapter}
  {\def\wilh@nomsection{section}}
  {\def\wilh@nomsection{chapter}}  
\def\thenomenclature{%
  \begin{multicols}{2}[%  
    \csname\wilh@nomsection\endcsname*{\nomname}
    \if@intoc\addcontentsline{toc}{\wilh@nomsection}{\nomname}\fi
    \nompreamble]
  \list{}{%
    \labelwidth\nom@tempdim
    \leftmargin\labelwidth
    \advance\leftmargin\labelsep
    \itemsep\nomitemsep
    \let\makelabel\nomlabel}%
}
\def\endthenomenclature{%
  \endlist
  \end{multicols}
  \nompostamble}
\makeatother

% % % % % % % % % % % % % % % % % % % % % % % % % % % % % % % % % % % %


% % % % % % % % % % % % % % % % % GLOSSARIES % % % % % % % % % % % % % % % % % % % % % % % % % % %
%Load the package: enables multiple glossaries
\usepackage[
nonumberlist, %do not show page numbers
acronym,      %generate acronym listing   -> Not used in this example (see line with %%% )
toc,          %show listings as entries in table of contents
section]      %use section level for toc entries
{glossaries}




%Generate a list of symbols
\newglossary[glg]{glossary}{gls}{glo}{Glossary}
\newglossary[alg]{acronyms}{acr}{acn}{Acronyms}
\newglossary[slg]{symbols}{syi}{syg}{Symbols}


%Load nomenclature and glossary files
\loadglsentries{glossary}
\loadglsentries{acronym}
\loadglsentries{symbols}


% Style for making glossary in two columns
\usepackage{glossary-mcols}
\newglossarystyle{glossary2col}{% 
	\glossarystyle{mcoltreenoname}% 
	\renewcommand*{\glossaryentryfield}[5]{% 
		\item[\glsentryitem{##1}\glstarget{##1}{##2}] \noindent##3 \hfill ##4 
	}% 
	\renewenvironment{theglossary}% 
	{% 
		\begin{multicols}{2} 
			\begin{description}[style=multiline,leftmargin=16mm,font=\normalfont] 
				\setlength{\parindent}{0pt}% 
				\setlength{\parskip}{0pt plus 0.0pt}% 
	}{ 
			\end{description} 
		\end{multicols} 
	}% 
}
% Style for no extra margin
\newglossarystyle{glossary1col}{% 
	\glossarystyle{mcoltreenoname}% 
	\renewcommand*{\glossaryentryfield}[5]{% 
		\item[\glsentryitem{##1}\glstarget{##1}{##2}] \noindent##3 \hfill ##4 
	}% 
	\renewenvironment{theglossary}% 
	{% 
		\begin{description}[style=multiline,leftmargin=13mm,font=\normalfont] 
			\setlength{\parindent}{0pt}% 
			\setlength{\parskip}{0pt plus 0.3pt}% 
	}{ 
		\end{description}  
	}% 
}
\newglossarystyle{single_coltree}{%
	\glossarystyle{tree}%
	\renewenvironment{theglossary}%
	{%
		\begin{multicols}{1}
			\setlength{\parindent}{0pt}%
			\setlength{\parskip}{0pt plus 0.3pt}%
		}%
		{\end{multicols}}%
}

%Remove the dot at the end of glossary descriptions
\renewcommand*{\glspostdescription}{}

%Activate glossary commands
\makeglossaries %

\setlength{\glsdescwidth}{0.9\linewidth}

%These commands sort the lists
%%%makeindex -s filename.ist -t filename.alg -o filename.acr filename.acn
%makeindex -s filename.ist -t filename.glg -o filename.gls filename.glo
%makeindex -s filename.ist -t filename.slg -o filename.syi filename.syg

% Make command to build the glossaries:
%pdflatex -synctex=1 -interaction=nonstopmode %.tex | makeglossaries % | pdflatex -synctex=1 -interaction=nonstopmode %.tex



\setlength{\columnsep}{4em}	% specifies the margin (space) between columns in the nomenclature
\setlength{\nomitemsep}{0.01cm} % specifies the space between item and explanation in the nomenclature

% Makes it possible to specify unit inside nomenclature environment with nomunit
\newcommand{\nomunit}[1]{%
	\renewcommand{\nomentryend}{\hspace*{\fill}#1}}


% % % % % % % % % % % % % % % % % % % % % % % % % % % % % % % % % % % % % % % % % % % % % % % %






%Appendixops�tning
\usepackage[toc,page,titletoc]{appendix}
\renewcommand{\appendixname}{Appendix}
\renewcommand{\appendixtocname}{Appendix}
\appendixpageoff
\appendixheaderon
\appendixtocoff
\appendixtitletocon
\appendixtitleon


% Enable page count between two labels
\usepackage{refcount}
\newcommand{\pagedifference}[2]{%
\number\numexpr\getpagerefnumber{#2}-\getpagerefnumber{#1}\relax}


%  Oversttelse og tegnstning 
\usepackage{t1enc}											% Hjlper med orddeling ved ,  og .
\usepackage[english]{babel} 							% Dansk sprog, bl.a. figure bliver til figur
%\usepackage[latin1]{inputenc}					% Gr det muligt at bruge ,  og  i sine .tex-filer
\usepackage[ansinew]{inputenc}					% Gr det muligt at bruge ,  og  i sine .tex-filer
\usepackage[T1]{fontenc}						% Giver flere mulige karakterer, s� fx � er �n karakter i stedet for to best�ende af o og �
\usepackage{latexsym}										% LaTeX symboler
\usepackage{ragged2e}										% Gr det mulig at venstre-/hjrecentrere blokke
\usepackage{import}					%For at importere svg med latex font
\usepackage{longtable}

%  Figurer og tabeller  floats  
%\usepackage{look/svg}	%makes it possible to define height of pdf_tex files for use in subfigure
\usepackage{graphicx} %til Matlab%
\usepackage{grffile}  %til Matlab%


%\usepackage{subfigure}               		% Til at stte flere underfigurer med hver sin caption ind i samme figur.
%\usepackage{subfig}
%\renewcommand{\thesubtable}{\arabic{subtable}}
%\captionsetup[subtable]{labelformat=empty}
\usepackage{sidecap} 										% Caption ved siden af figurer / tabellen.




\usepackage{flafter}										% Srger for at dine floats ikke optrderi texten fr de er sat ind.
\usepackage{tabularx}										% Muliggr at angive bredden af tabellen.
\newcolumntype{L}{>{\raggedright\arraybackslash}X}			% venstrejustering i tabularx (L=X)
\newcolumntype{R}{>{\raggedleft\arraybackslash}X}			% G�r det muligt at h�jrejustere i tabularx
\usepackage{float}											% Forbedrer samspillet mellem defineringer af flydende objekter.
\usepackage{longtable}									% Gr s tabeller lettere kan strkke sig over flere sider

\usepackage{multirow}                   % Tabelfunktion
\usepackage{hhline}                     % Tabelfunktion
\usepackage{multicol}                   % Tabelfunktion
\usepackage{wrapfig}										% Muliggr float med figurer



%  Matematiske formler og maskinkode 
\usepackage{amssymb}										% Giver mulighed for matematiske symboler.
%\usepackage[fleqn]{amsmath}
\usepackage{amsmath}										% Matematisk pakke
\usepackage{mathtools}									% Udvidelse af amsmath-pakken.
\allowdisplaybreaks 
%\usepackage{theorem}										% Matematisk pakke - skal udkommenteres for at example-environment (ntheorem) fungerer
\usepackage{listings}										% Linie nummerering af programkode
\usepackage{color}                        % Package til \color-kommandoen

\usepackage{soul}
\usepackage{courier}

%\newcommand{\pp}{\begin{flalign}}
\newcommand{\kk}{\hspace{0.8cm}}
\newcommand{\mm}{\hspace{0.3cm}}

%\usepackage{xcolor}						% indeholder 68 pr�definerede farver
%\usepackage{listings}                     % Package til kodeeksempler
\def\lstlistingname{Code snip}%        % Definerer hvad der str foran et stykke kodes caption
\lstset{
	basicstyle=\small\ttfamily,                    % Lille skrifttype
	keywordstyle=\color{blue}\bfseries,   % Keywords bl og bold
	commentstyle=\color[RGB]{126,126,126},  % Comments svag sort
	showstringspaces=false,               % Ingen symbol for mellemrum i strings
	numbers=left,                         % Linjenumre til venstre
	numberstyle=\tiny,                    % Sm tal p linjenumre
	numbersep=5pt,                        % ?
	tabsize=4,                            % Indenteringer = 4 spaces
	columns=flexible,                     % Font width = low
	breaklines=true,                      % Deler en for lang linje over to linjer
	frame=leftline,                       % Streg til venstre, der afgrnser kode
	captionpos=b,                         % Caption til kode under kodeeksemplet
	escapeinside={(*@}{@*)}               % Giver mulighed for at lave en (*@\label{label}@*), p en kodelinje,
%										    s man kan referere til linjen
}
\usepackage{setspace}					%Giver adgang til onehalfspacing
\onehalfspacing     					%S der er halvanden linjeafstnad
\usepackage{textcomp}  % g�r det muligt at skrive euro-tegn (?)

\definecolor{matlab-color}{rgb}{0.13333,0.545,0.13333}
\definecolor{matlab-color2}{rgb}{0.627,0.125,0.94}

\lstdefinelanguage{matlab} {              % Definition af Arduino-language
	morekeywords={all, on, quiet, x(n,N), u(p,N),to},
		keywordstyle=\color[RGB]{160,32,240},classoffset=2, %purple
			morekeywords={end, for, if, function, elseif, else},
		keywordstyle=\color[RGB]{0,0,255},classoffset=3, %blue
%   sensitive=false, morecomment=[l]{\%}, morecomment=[s]{/*}{*/} % Ingen farver p kommentare
	 comment=[l][\color{matlab-color}]{\%},%morecomment=[s]{'}{'} % Ingen farver p kommentare
%	 morecomment=[s][\color{matlab-color2}]{'}{'}
	%	keywordstyle=\color[RGB]{34,139,34},classoffset=3,
}



\lstdefinelanguage{Arduino} {              % Definition af Arduino-language
	morekeywords={pinMode, digitalWrite, begin, analogRead, delay, delayMicroseconds, if, println, print, for, int, float, char, volatile, micros, do, return, analogReference, analogWrite},
		keywordstyle=\color[RGB]{204,102,0},classoffset=2,
	morekeywords={OUTPUT, HIGH, LOW, INPUT, DEFAULT},
		keywordstyle=\color[RGB]{0,102,153},classoffset=1,
	morekeywords={=, \&, <, >, +, -, *, /, ., ||},
		keywordstyle=\color[RGB]{0,0,0},classoffset=0,
	morekeywords={loop, setup, serial, while, void, double, unsigned, long, int, float, uint_8},
		keywordstyle=\color[RGB]{204,102,0},classoffset=0,
	sensitive=false, morecomment=[l]{//}, morecomment=[s]{/*}{*/} % Ingen farver p kommentare
}
\lstdefinelanguage{Assembler} {              % Definition af Assembler-language
basicstyle=\ttfamily\scriptsize,
	morekeywords={DSIN, DSOUT, EQU},
		keywordstyle=\color[RGB]{15,250,15},classoffset=4,
%		
	morekeywords={	ADD, ADDH, B, BC, BCND, BLZ, CALL ,CC , CRLT, RETD, RETE, RET, RETI, ADDCY, AND,CLRC,SETC, CALL, COMP, DINT, EINT, FETCH, IN, JUMP, LOAD, OR, OUT, ENABLE, DISABLE, APAC, SACH, MAC, NOP, RPTZ, LAR, LACL, MAR, RPT, SACL, SPLK, LT, MPY, LTD, LTA, SFL, ABS, SAMM, LAMM, LDP, RL, RR, SACB, SL0, SL1, SLA, SLX, SR0, SR1, SAR, SRA, SRX, STORE, SUB, TEST, XOR, LALK, SUBK, BNZ, PAC },
		keywordstyle=\color[RGB]{20,10,150},classoffset=3,
%		
	morekeywords={XF,EQ	\$00, \$01,\$1, \$02, \$03, \$04, \$05, \$06, \$07, \$08, \$09, \$0A, \$0B, \$0C, \$0D, \$0E, \$0F,
					\$10, \$11, \$12, \$13, \$14, \$15, \$16, \$17, \$18, \$19, \$1A, \$1B, \$1C, \$1D, \$1E, \$1F,
					\$20, \$21, \$22, \$23, \$24, \$25, \$26, \$27, \$28, \$29, \$2A, \$2B, \$2C, \$2D, \$2E, \$2F,
					\$30, \$31, \$32, \$33, \$34, \$35, \$36, \$37, \$38, \$39, \$3A, \$3B, \$3C, \$3D, \$3E, \$3F,
					\$40, \$41, \$42, \$43, \$44, \$45, \$46, \$47, \$48, \$49, \$4A, \$4B, \$4C, \$4D, \$4E, \$4F,
					\$50, \$51, \$52, \$53, \$54, \$55, \$56, \$57, \$58, \$59, \$5A, \$5B, \$5C, \$5D, \$5E, \$5F,
					\$60, \$61, \$62, \$63, \$64, \$65, \$66, \$67, \$68, \$69, \$6A, \$6B, \$6C, \$6D, \$6E, \$6F,
					\$70, \$71, \$72, \$73, \$74, \$75, \$76, \$77, \$78, \$79, \$7A, \$7B, \$7C, \$7D, \$7E, \$7F,
					\$80, \$81, \$82, \$83, \$84, \$85, \$86, \$87, \$88, \$89, \$8A, \$8B, \$8C, \$8D, \$8E, \$8F,
					\$90, \$91, \$92, \$93, \$94, \$95, \$96, \$97, \$98, \$99, \$9A, \$9B, \$9C, \$AD, \$9E, \$9F,
					\$A0, \$A1, \$A2, \$A3, \$A4, \$A5, \$A6, \$A7, \$A8, \$A9, \$AA, \$AB, \$AC, \$BD, \$AE, \$AF,
					\$B0, \$B1, \$B2, \$B3, \$B4, \$B5, \$B6, \$B7, \$B8, \$B9, \$BA, \$BB, \$BC, \$CD, \$BE, \$BF,
					\$C0, \$C1, \$C2, \$C3, \$C4, \$C5, \$C6, \$C7, \$C8, \$C9, \$CA, \$CB, \$CC, \$CD, \$CE, \$CF,
					\$D0, \$D1, \$D2, \$D3, \$D4, \$D5, \$D6, \$D7, \$D8, \$D9, \$DA, \$DB, \$DC, \$DD, \$DE, \$DF,
					\$E0, \$E1, \$E2, \$E3, \$E4, \$E5, \$E6, \$E7, \$E8, \$E9, \$EA, \$EB, \$EC, \$ED, \$EE, \$EF,
					\$F0, \$F1, \$F2, \$F3, \$F4, \$F5, \$F6, \$F7, \$F8, \$F9, \$FA, \$FB, \$FC, \$FD, \$FE, \$FF
					},
		keywordstyle=\color[RGB]{150,120,10},classoffset=2,
%		
	morekeywords={AR0, AR1, AR2, AR3, AR4, AR5, AR6, AR7, DRR, DXR},
		keywordstyle=\color[RGB]{75,0,0},classoffset=1,
%		
	sensitive=false, morecomment=[l]{;} % Ingen farver p kommentare
}
\lstdefinelanguage{VHDL} {              % Definition af VHDL-language
basicstyle=\ttfamily\scriptsize,
	morekeywords={IEEE, STD_LOGIC_1164, NUMERIC_STD, STD_LOGIC_VECTOR, STD_LOGIC, STD_LOGIC_ARITH, STD_LOGIC_UNSIGNED, rising_edge},
		keywordstyle=\color[RGB]{150,0,10},classoffset=4,
%		
	morekeywords={library, use, ALL, entity, is, Port, in, out, downto, end, begin, 
				 architecture, of, signal, map, variable, integer, 
				 process, when, if, elsif, else, others, with, select, then, case, 
				 NOT, AND, OR, XOR },
		keywordstyle=\color[RGB]{0,0,250},classoffset=3,
		%
    morekeywords={0, 1, 45, 152 },
		keywordstyle=\color[RGB]{255,40,0},classoffset=2,
%		
%	morekeywords={	},
%		keywordstyle=\color[RGB]{250,210,0},classoffset=2,
%		
	sensitive=false, morecomment=[l]{--} % Ingen farver p kommentare
}
\lstdefinelanguage{constraint} {              % Definition af .ucf-language
basicstyle=\ttfamily\scriptsize,
	morekeywords={FALSE, TRUE},
		keywordstyle=\color[RGB]{150,0,10},classoffset=4,
%		
	morekeywords={NET, LOC, PERIOD},
		keywordstyle=\color[RGB]{0,0,250},classoffset=3,
%		
%	morekeywords={	},
%		keywordstyle=\color[RGB]{250,210,0},classoffset=2,
%		
	sensitive=false, morecomment=[l]{\#} % Ingen farver p kommentare
}

% Eksempler
\usepackage{framed}%
%%\usepackage{amsmath}
\usepackage[framed,amsmath,thmmarks]{ntheorem}%	% when using thmmarks, amsmath must be an option as well. Otherwise \eqref doesn't work anymore.
\usepackage{aliascnt}%
\usepackage{hyperref}%							% Giver mulighed for at ens referencer bliver til en slags hyperlinks.
\theoremheaderfont{\normalfont\bfseries}% 		% bruger samme font som resten af rapporten, i bold
\theorembodyfont{\normalfont}% 					% bruger samme font som resten af rapporten
\theoremstyle{break}% 							% linieskift efter overskrift
\def\theoremframecommand{{\color{HeaderBlue}\vrule width 10pt \hspace{8pt}}}%
\newtheorem{theorem}{Theorem}%
\newaliascnt{example}{theorem}%
\newshadedtheorem{exa}{Definition}[chapter]%		% navngivning af eksempel
\aliascntresetthe{example}%
\providecommand*{\exaautorefname}{eksempel}%	% G�r det muligt at bruge autoref p� eksempler, og s� er de navngivet Eksempel #.#
\newenvironment{example}[1]{%					% tager �t input i {}-klammer (titel p� eksempel)
	\begin{exa}[#1]%
}{%
	\end{exa}%
}
%\newcommand{\exref}[1]{Eksempel~\ref{#1}}		% definerer hvordan der kan refereres til et eksempel s�ledes at referencen kommer til at hedder "Eksempel #" i stedet for "Titel #", hvor Titel er den titel der er givet eksemplet, angivet i lin 124 (5 linjer f�r denne). inde i environmentet skal \label s�ttes i linjen efter \begin{example}{Titel}, og n�r der skal refereres til eksemplet anvendes \exref{label_p�_eksemplet}


%\usepackage{xfrac}                                         %lort

%  PDF og billede optimering 
\usepackage{pslatex} 								 		% Pnere PDF-filer
\usepackage{pdfpages}	
\pdfoptionpdfminorversion=6		
%\usepackage[pdftex]{graphicx} 					% Pakke til jpeg/png billeder
\usepackage{graphicx}                		% Pnere grafik
\usepackage[dvips]{epsfig}           		% For at inkludere eps billeder
\usepackage{epstopdf}

%  Refrenncer, litteraturliste og URL'er 
\usepackage{url}												% Til at stte URL'er op med. Virker sammen med hyperref
\usepackage{fancyref}										% Srger for at referencerne fr de rigtige ord med p vejen.
\usepackage{varioref}										% Includerer sidenummeret i krydsreferancerne. Ikke hvis det er p samme side som referencen.
\usepackage{natbib}				% Gr det muligt at bruge en rkke forskellige citationsmetoder, f.eks. Harvard
\usepackage[draft,footnote,nomargin]{fixme}


%% \Autoref is for the beginning of the sentence
%\let\orgautoref\autoref
%\providecommand{\Autoref}[1]%
%{%
%\def\equationautorefname{Equation}%
%\def\figureautorefname{Figure}%
%\def\subfigureautorefname{Figure}%
%\def\chapterautorefname{chapter}%
%\def\tableautorefname{Table}%
%\def\sectionautorefname{Sect.}%
%%\def\appendixautorefname{Appendix}
%%
%  \def\footnoteautorefname{footnote}%
%  \def\itemautorefname{item}%
%  \def\partautorefname{Part}%
%  \def\subsectionautorefname{subsection}%
%  \def\subsubsectionautorefname{subsubsection}%
%%  \def\paragraphautorefname{paragraph}%
%%  \def\subparagraphautorefname{subparagraph}%
%  \def\FancyVerbLineautorefname{line}%
%%  \def\theoremautorefname{Theorem}%
%  \def\pageautorefname{page}%
%%
%%\def\appendixautorefname{Appendix}
%\orgautoref{#1}%
%}

% \Autoref is for the beginning of the sentence
\let\orgautoref\autoref
\providecommand{\Autoref}[1]%
{%
\def\exaautorefname{Definition}%
\def\equationautorefname{Equation}%
\def\figureautorefname{Figure}%
\def\subfigureautorefname{Figure}%
\def\chapterautorefname{Chapter}%
\def\tableautorefname{Table}%
\def\sectionautorefname{Section}%
\def\appendixautorefname{Appendix}%
%
\def\footnoteautorefname{Footnote}%
\def\itemautorefname{Item}%
\def\partautorefname{Part}%
\def\subsectionautorefname{Subsection}%
\def\subsubsectionautorefname{Subsection}%
%  \def\paragraphautorefname{paragraph}%
%  \def\subparagraphautorefname{subparagraph}%
\def\FancyVerbLineautorefname{Line}%
%  \def\theoremautorefname{Theorem}%
\def\pageautorefname{Page}%
\def\lstlistingautorefname{Code snip}%
\def\exaautorefname{Definition}%
%\def\appendixautorefname{Appendix}
\orgautoref{#1}%
}




% \autoref is used inside the sentence to produce Fig., and Eq. for figures, subfigures, and equations
\renewcommand{\autoref}[1]%
{%
\def\equationautorefname{equation}%
\def\figureautorefname{figure}%
\def\subfigureautorefname{figure}%
\def\chapterautorefname{chapter}%
\def\tableautorefname{table}%
\def\sectionautorefname{section}%
\def\appendixautorefname{appendix}%
%
\def\footnoteautorefname{footnote}%
\def\itemautorefname{item}%
\def\partautorefname{part}%
\def\subsectionautorefname{subsection}%
\def\subsubsectionautorefname{subsection}%
%  \def\paragraphautorefname{paragraph}%
%  \def\subparagraphautorefname{subparagraph}%
\def\FancyVerbLineautorefname{line}%
\def\theoremautorefname{theorem}%
\def\pageautorefname{page}%
\def\lstlistingautorefname{code snip}%
\def\exaautorefname{definition}%
\orgautoref{#1}%
}




%  Andre smarte pakker  
\usepackage{ifthen}											% Tillader brugen af bolske ligniger, if, then osv
\usepackage{soul} 											% Understtter understregning af tekst ved at skrive \ul{tekst}
\hypersetup{pdfborder = 0} 							% Fjerner ramme omkring links i fx indholsfortegnelsen og ved kildehenvisninger
\usepackage{lastpage}									% Til hvis man har brug for det totale antal sider.




\lstset{basicstyle=\scriptsize\sf, tabsize=3, numberstyle=\tiny, stepnumber=1, numbersep=5pt,  language={java}, extendedchars=true} % Fjerner columns=flexible, samt numbers=left 


\addtolength{\topmargin}{-1cm}					% Tilfjer vrdier til de respektive parametre
\addtolength{\textheight}{2cm}					% -
\addtolength{\textwidth}{2.0cm}   			% -
\addtolength{\oddsidemargin}{7mm} 			% -
\addtolength{\marginparwidth}{-1cm} 		% -
\addtolength{\evensidemargin}{-2.3cm} 	% -

\setlength{\parindent}{0mm}          		% Ingen indryk
\setlength{\parskip}{1mm}          			% Afstand mellem afsnit
\linespread{1.1}                     		% Linie afstand
\setcounter{tocdepth}{1} 								% Angiver dybden i indholdsfortegnelse (1 angiver til og med \section).     

% Kapiteludseende. De udkommenterede er andre typer.

%%%%%%%% Overskrift 1 %%%%%%%%%%
%\usepackage{curves}             				% Used by jalchap
%\usepackage{look/jalchap}        				% The chapter heading

%%%%%%%% Overskrift 2 %%%%%%%%%%
%\usepackage{look/lgchappng}

%%%%%%%% Overskrift 3 %%%%%%%%%%
%\usepackage{look/popsmart}

%%%%%%%% Overskrift 4 %%%%%%%%%% 
%(Fjern alle %-tegn -->)

\addtolength{\hoffset}{-1.6cm}
\setlength{\textwidth}{16.0cm}
\setlength{\topmargin}{-0.4in}
\setlength{\topskip}{0.2in}
\setlength{\textheight}{9.4in}
\setlength{\oddsidemargin}{2.00cm}
\setlength{\evensidemargin}{1.00cm}
\makeatletter 
\def\@makechapterhead#1{
   \vspace *{-15mm} % 15                         % afstand fra topmargen til
\hskip -5mm
  \rule[4.8mm]{120mm}{0.3mm}             % verste bjlke
  \hskip 5mm 
  \huge                                   % formattering af "Kapitel"
  {\centering \textbf{\@chapapp{} \thechapter}}
{ 
  \vskip 0 mm                             % afstand mellem topstreg og kaptxt
 \parbox{150mm}{{\bf{\Huge  #1}}}\par     % kaptxt
  \vskip 5mm                             % afstand mellem kaptxt og bundstreg
 \ifnum
  \c@secnumdepth >\m@ne 
  }
  \vskip 0mm    % 0                          % afstand mellem txt og bundbjlke
\hskip -5mm % -5
   \rule[15mm]{17cm}{0.3mm}   % 15,17,0.3            
   \par\nobreak\normalsize
  \fi  
}
%%%% (<-- Hertil)


%%%% Fancy headings & chapters %%%%
\usepackage{fancyhdr}
\pagestyle{fancyplain}
\renewcommand{\footrulewidth}{0.4pt}
\renewcommand{\chaptermark}[1]{\markboth{\thechapter\ #1}{}}
\renewcommand{\sectionmark}[1]{\markright{\thesection\ #1}}
\pagestyle{fancy}
\fancyhead{}                            % Sletter alt nuvrende hoved- og fodkonfiguration
\fancyhead[RO,LE]{\nouppercase \slshape \rightmark}
\fancyfoot[LE,RO]{\thepage}
\fancyfoot[C]{ }
\fancyfoot[RE,LO]{\emph{\leftmark}}
\renewcommand{\headrulewidth}{0.4pt}    % Bredde af hovedets streg
\renewcommand{\footrulewidth}{0.4pt}    % Bredde af fodens streg
% ***** Layout for sider med \chapter eller \thispagestyle m.fl. *****
\fancypagestyle{plain}{
\fancyhf{}                              % Sletter alt nuvrende hoved- og fodkonfiguration
%\fancyfoot[C]{\thepage\\ \tiny{Kompileret \today { }kl. \printtime { }af \input{userfile}}}
\fancyfoot[LE,RO]{\thepage}
\renewcommand{\headrulewidth}{0.0pt}    % Bredde af hovedets streg
\renewcommand{\footrulewidth}{0.0pt}    % Bredde af fodens streg
}
% *****
\newcommand{\celcius}{^{\circ}C}
\newcommand{\bmath}[0]{\begin{eqnarray}}
\newcommand{\emath}[0]{\end{eqnarray}}

\bibpunct[,]{[}{]}{;}{a}{,}{,} % Definerer de 6 parametre ved Harvard henvisning (bl.a. parantestype og seperatortegn)

\bibliographystyle{look/plainnat-custom} 						% Udseende af litteraturlisten

% Settings for bibliography and table of contents
%\bibliographystyle{apalike}
%\bibliographystyle{alpha}
%\setcounter{tocdepth}{1} %Bestemmer antallet af undersektioner der skal med i indholdsfortegnelsen
\renewcommand{\chaptermark}[1]{%
\markboth{\thechapter.\ #1}{}}
\newcommand{\clearemptydoublepage}{\newpage{\pagestyle{empty}\cleardoublepage}}

\parskip        =    1ex
\parindent      =    0em
\baselineskip   =    2ex

%%%%%%% TLLER TIL KRAVSPEK %%%%%%%%%%%%
\newcounter{kravenum}
\makeatletter{
\renewcommand\appendix{
\renewcommand\section{
\newpage\thispagestyle{plain}
\suppressfloats[t]\@afterindentfalse}
\secdef\Appendix\sAppendix}
\setcounter{section}{0}\renewcommand\thesection{Alph{section}}}
\newcommand\Appendix[2][?]{
\refstepcounter{section}
\addcontentsline{toc}{appendix}{\parskip=0ex}
{\protect\numerline{\appendixname~\thesection}#1}
{\raggedleft\large\bfseries \appendixname\
\thesection\par \centering#2\par}
\sectionmark{#1}
\@afterheading
\addvspace{\baselineskip}}
\newcommand\sAppendix[1]{
{\raggedleft\large\bfseries\appendixname\par \centering#1\par}
\@afterheading\addvspace{\baselineskip}}
\makeatother
\newcommand*{\journal}{
        \renewcommand{\thechapter}{\Roman{chapter}}    % Numeriske kapitelindeks
        \renewcommand{\chaptername}{Bilag}             % "Bilag" som kapitelnavn
        \setcounter{chapter}{0}
        \rfoot{\fancyplain{}{\textbf{Group 451}}}
				\cfoot{\fancyplain{}{\textbf{\thepage{}}}}
				\lfoot{\fancyplain{}{\textbf{Computer Engineering, AAU}}}
        }
\usepackage{color}											% Pakke til farvning af tekst med \textcolor{farve}{target}
\usepackage{colortbl}										
\usepackage{caption}										% Tillader brug af figurtekst
%\usepackage{ccaption}									% Kan stte caption under ting der ikke normalt har en caption.
\usepackage[neverdecrease]{paralist}
\usepackage{booktabs}

\captionsetup{													% Figurtekst opstning
font=small,															% Skriftstrrelse
labelfont={bf,it},											% Figurlabel skrives altid med fed og kursiv
margin=20pt,														% Definerer margin
format=hang,														% Figurtekst p flere linier ordnes under hinanden og ikke helt ind under "Figur"
}

\let\olditemize=\itemize								% Fjerner den vertikale afstand mellem punktopstillinger
\def\itemize{
        \olditemize
        \setlength{\itemsep}{-1ex}
        }
\let\oldenumerate=\enumerate						% Fjerner den vertikale afstand mellem listeopstillinger
\def\enumerate{
        \oldenumerate
        \setlength{\itemsep}{-1ex}
        }

\hyphenation{hvis hvor hvad MPa ind-frt fle-re hvor-ved pro-jekt-pe-ri-o-dens In-ge-ni-r}

%Colours for P4
\definecolor{LightSkyBlue}{rgb}{0.529,0.8078,0.980}
\definecolor{LighterSkyBlue}{rgb}{0.678, 0.847,0.961}
\definecolor{LightererSkyBlue}{rgb}{0.957, 1, 1}
\definecolor{HeaderBlue}{rgb}{0.80, 0.87, 1}    % {0.85, 0.925, 1}
\definecolor{textBlue}{rgb}{0.90, 0.94, 1}    % {0.95, 0.98, 1}
\definecolor{white}{rgb}{1, 1, 1}
\definecolor{grey}{rgb}{0.7, 0.7, 0.7}
\definecolor{orange}{rgb}{1,0.4,0}
\definecolor{OliveGreen}{rgb}{0.14,0.6,0.3}
\definecolor{MidnightBlue}{rgb}{0.416,0.3529,0.804}
\definecolor{Chocolate}{rgb}{0.804,0.41176,0.1176}
\definecolor{WildStrawberry}{rgb}{1,0.263,0.643}
%\definecolor{Mid-aubergine}{hex}{#5E2750}
\usepackage[usenames,dvipsnames]{xcolor}



%Colours for P5
\definecolor{HeaderGreen}{rgb}{0.796, 0.906, 0.843}	% 203 231 215 %OLD 185 213 167
\definecolor{textGreen}{rgb}{0.922, 0.965, 0.937}	% 235 246 239 %OLD 206 229 185
%\definecolor{textGreenLight}{rgb}{0.925, 0.957, 0.878}    % 236 244 224

\usepackage{enumerate}
\usepackage{enumitem}



%\usepackage{showframe}

\RequirePackage[caption=false,position=bottom]{subfig}
\let\subbottom\subfloat

%\usepackage{oubraces} % used as: \overunderbraces{upper_braces}{main_formula}{lower_braces} for example \overunderbraces{&\br{num_columns}{text_above}}% upper brace
%{a+b+&c+d+&e+f+g&+h+i}% main formula
%{&&\br{num_columns}{text_below}} % lower brace

\usepackage{dirtree}
\usepackage{qtree}
% Like \DescribeMacro, but arbitrary text
\def\marginpost#1{\leavevmode \marginpar{~\hfill\texttt{#1}}\ignorespaces}

% The name of the package, capitalized or not.
\def\QT{\textsf{\emph{Qtree}}}
\def\Qt{\textsf{\emph{qtree}}}

\def\pstyle#1{\textsf{#1}} % style filenames

\def\argplain#1{\textit{$\langle$#1$\rangle$}} % Like \marg, but without { }
\usepackage{graphicx}
\usepackage{tikz-qtree}

\usepackage{tikz}
%%%<
\usepackage{verbatim}
%\usepackage[active,tightpage]{preview}
%\PreviewEnvironment{tikzpicture}
%\setlength\PreviewBorder{5pt}%
\usetikzlibrary{arrows}
\usetikzlibrary{trees,calc}