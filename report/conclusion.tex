This chapter will conclude on the results obtained throughout this thesis and put the solution and entire strategy into perspective in the discussion part.
\section*{Conclusion}
Safety aspects in robotic surgery and automated robotic surgery are found to be \textit{the} important factor, as analysed in \autoref{chap:intro}. Concurrently, it founds the basic framework in the long term goal of obtaining virtual fixtures. Consequently, a barrier certificate is stated in \autoref{chap:barrier_cerificates} which modifies and adapts the Lyapunov stability criteria to enable a way to define  safe and unsafe regions within the state-space. 

A theoretical controller is developed in \autoref{chap:cbf} based on control barrier functions which ensures that the  barrier certificate requirements presented in \autoref{chap:barrier_cerificates} are obeyed at all times. Thus, the control barrier function allows a way to ensure safety in real-time with astounding few calculations.

The control topology presented in \autoref{chap:cbf} is applied to three use cases which intend to commence a solution to the problem of guaranteeing safety in automated surgeries, i.e.:
\begin{itemize}
\item A concrete example of the use of control barrier functions is founded in \autoref{chap:cbf_1d_static}. It comprises the instrument slide movement. The system is modelled as both a first and second order system, thereby slowly increasing the complexity of the CBFs, such that necessary experience in the construction of CBFs can be gathered. The result is a successful controller guaranteeing safety by never entering predefined unsafe regions in one dimension for both a first and second order system approximation.
\item A safe regulator is designed to ensure system safety in relation to virtual fixtures in \autoref{chap:cbf_1d_dynamic}. A dynamic CBF is  constructed in accordance with the desire of virtual fixture and thus founds safety for operation on a beating heart. The result for this use case is that a safe distance between heart and robotic end effector can be set as desired.
\item Then safety considerations are extended to the 3D Euclidean space in \autoref{chap:cbf_3d_static} which implies additional implementation challenges such as a kinematic description (mapped and verified in \autoref{app:kinematic_model_robot}), forward kinematics and inverse kinematics. The construction of a CBF is taken to higher dimensions forming a barrier enclosing the interior of an ellipsoid, thus representing a heart or another vital organ fixed in space. The result is a valid CBF ensuring that the robot end effector is kept outside the ellipsoid at all time.
\end{itemize}
All three use cases are implemented in a simulation environment in MATLAB with convincing results, i.e. system safety is ensured by preventing the system state from entering specified unsafe regions. The controllers are furthermore implemented in C++ in the ROS (Robotic Operating System) framework. The ROS framework is founded in \autoref{app:ros} as a necessary condition to allow any implementation on the da Vinci robot. All development within ROS is tailored for this project and did not exist at project initiation. The implemented controllers comply with the expected outcome and do indeed behave as desired, i.e. ensuring safety by evading the predefined unsafe regions. Additionally, the implemented controllers are verified to require very little processing power making them ideal as real-time controllers.

The three use cases do, however, consist of simple models where the system order does not exceed 3. An important conclusion is drawn from the use cases, which already could be inferred from the one dimensional safe slide controller (developed in \autoref{chap:cbf_1d_static}) with system order 2. That is, for high order systems where the physical interpretation of the state vector is obscured, the construction of a valid CBF is a highly non-trivial task -- if not impossible. 

For this reason, the problem is turned upside down in \autoref{chap:putinar}, thus no restrictions are put forth in the controller development. Instead, the closed loop system is evaluated and the question is asked whether it complies with the barrier certificate requirement in \autoref{chap:barrier_cerificates}. The verdict is hereafter given as \textit{pass} or \textit{not pass}. For this purpose, \autoref{chap:putinar} presents the global SOS (Sum Of Squares) positivity characteristic and through Putinar's Positivstellensatz recast the barrier certificate formulation as a  problem of local positivity, thus allowing sets of unsafe and safe regions to be defined by unrestricted polynomials.

The strategy presented in \autoref{chap:putinar} is applied with the MATLAB toolbox SOSTOOLS in \autoref{chap:sostools} such that  barrier certificates can be searched for by automated means. Here, a framework is developed such that the toolbox takes a closed loop system description and a description of the safe and unsafe regions as inputs. The developed framework delivers an unambiguous certificate answering if the system is safe, thus constituting the \textit{pass} and \textit{not pass} verdict. The slide controller developed in \autoref{chap:cbf_3d_static} is accordingly taken as an example and the framework is verified with this example. Both the first and second order system approximation is analysed in the designed SOSTOOLS framework. It is, as expected, certified to be safe in almost the entire desired range.  These examples conclude and verify the use of the developed framework. The framework can easily handle other systems, as the task merely comprises other closed loop system descriptions as input in other dimensions with different safe and unsafe sets. This is a trivial task.

Hence, it can be concluded that the two initially desired strategies comprising the design and analysis of a safe controller are investigated and solved sufficiently to provide a "proof of concept" framework. This applies for both theory, simulation and implementation.


\section*{Discussion and Future Work}
The developed solution proves itself very efficient in both theory and simulation. However, the implementation aspect suffers from a number of issues which should be investigated in future work. This includes:
\begin{itemize}
\item Incorporate integral action in all controllers to eliminate steady state errors.
\item Increase the sampling rate from 100\,Hz to 2\,kHz which indeed is the long term goal. All controllers will draw benefits from this on the transition set $\mathcal{T}$. This may, however, introduce challenges as the allowed execution time (process time between every sample) is lowered to 0.5\,ms which is less than the actual execution time in \autoref{fig:3d_exe} for the safety controller in the 3D Euclidean space. Therefore, optimization must be performed in the implementation.
\item Improvement of the inverse kinematics solver as it occasionally chooses joint angles requiring multiple revolutions around the unit circle to obtain a position which could be reached with an angle less than $\pi$.
\end{itemize}
Additionally, the position controller already implemented on the FPGA (as seen in \autoref{fig:overview}), is left untouched. It may with removed to draw benefits from a more clear dynamics. This will require another system model, but may well be worth the trouble.

The application of the safety controllers derived throughout the project is exclusively the surgical robot. However, it should take very little imagination to envisage that this way of constructing controllers has the potential to be used in many other industries where safety is critical or simply where regions are desirable to be left untouched.

Furthermore, a consistently disregarded topic in this project is the use of trajectory planning. The controllers developed take only small steps as input. However, large step sizes have been given to the controllers in this project to demonstrate certain features, but obviously, it is desired to construct a trajectory planning layer taking the setpoints as input and breaking the path down into a sequence of adjacent points, thus ensuring that small step sizes are given to the controller.

Additionally, at no point  the orientation of the robot hand has been considered. Obviously, ensuring safety for the end effector is not sufficient as the heart or other vital organs can be penetrated or crushed by collision with the physical volume of the robotic tool other than the tip of the tool. This is an important topic in future work. Collision avoidance for the robotic parts themselves must also be studied when employing all four of the da Vinci arms in the setup, which is indeed the long term objective.

As explained by assistant nurse Jane Petersson in \autoref{sec:aau_doc}, there are veins, nerves and other organs which must not be cut during a surgery. It has been the aim to construct barrier certificates that can represent these parts of the body. However, it is clear that a realistic barrier certificate representing these parts is far away. Especially because they are time dependent and because, from time to time, the surgeon needs to move these parts to be able to operate in a certain area, thus reshaping these parts. Consequently, a very creative and adaptive barrier function is required and will as a necessary condition require robot vision (a continuous video stream analysis) such that these parts can be tracked. A way to resolve this complex problem of high dimensionality could be a combination of the design approach and analysis approach in the following way:
\begin{enumerate}
\item Search for a barrier certificate using the framework developed in \autoref{chap:sostools}.
\item Apply this barrier certificate as control barrier function in a similar way as done in \autoref{chap:cbf_1d_static}, \autoref{chap:cbf_1d_dynamic} and \autoref{chap:cbf_3d_static}.
\item Analyse the situation. Adjust the barrier certificate if necessary and describe the new closed loop system.
\item Take the new closed loop system as input to the framework developed in \autoref{chap:sostools} and start from 1 again.
\end{enumerate}
This iterative approach may along with the preceding listed bullet points get the robotic surgery industry one step closer to the end goal of guaranteed safe robotic surgery with the da Vinci robot. 

