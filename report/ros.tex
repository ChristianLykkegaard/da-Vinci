\lstdefinestyle{ubuntu}
{
    backgroundcolor=\color{black},
    basicstyle=\scriptsize\color{green}%\ttfamily
}
%
\chapter{Interfacing da Vinci with ROS}
This appendix ought to give any user concrete knowlegde to utilize the \gls{ros} environment wrt. the \gls{daVinci} surgery robot at Aalborg University as it comprises an immense load of files, packages and various GUI interfaces. The \gls{ros} environment is currently only developed for Ubuntu. The content of this appendix is accordingly assuming Ubuntu as operating system. 

To communicate with da Vinci, it is important to execute a few commands in the correct order. It is recommended to work directly on the server in the lab (it provides additional GUI applications such as rviz), but connection to a private laptop can be established through \texttt{ssh}:

%\begin{lstlisting}[style=ubuntu]
\hspace{1cm} \texttt{\$ ssh <user>@surgery-srv.lab.es.aau.dk}
%\end{lstlisting}

It is first of all important to collect all \gls{node}s such that they are able to communicate with each other. Open a terminal an run:

\hspace{1cm} \textbf{1.} \ \ \ \texttt{\$ roscore} \ \ \ {\color{RoyalBlue}{\textit{\# Leave this running in the terminal}}}

Now, to secure TCP/IP connection between ROS and the RIO board, launch the driver from a new terminal:

\hspace{1cm} \textbf{2.} \ \ \  \texttt{\$ roslaunch davinci\_driver/ davinci\_driver.launch} \ \ \ {\color{RoyalBlue}{\textit{\# Leave this running}}} 

To establish ???

\hspace{1cm} \textbf{3.a} \ \ \  \texttt{\$ roslaunch davinci\_moveit\_config/ move\_group.launch} \ \ \ {\color{RoyalBlue}{\textit{\# Leave this running}}} 

If a 3D GUI interface is desired, open a new terminal and launch:

\hspace{1cm} \textbf{3.b} \ \ \  \texttt{\$ roslaunch davinci\_bringup/ visualization.launch} \ \ \ {\color{RoyalBlue}{\textit{\# This opens rviz}}} 