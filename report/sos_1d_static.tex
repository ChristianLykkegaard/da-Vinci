In this and the following chapters, barrier certificates are attempted to be constructed by use of Putinar's Positivstellensatz in \autoref{def:putinar} with the Matlab toolbox SOSTOOLS (see \autoref{app:sostools} for a short introduction to the toolbox). In this chapter the first and second order systems depicted in \autoref{fig:stepresponseslide} are considered with a static reference on the position.




\section{Safety Verification of First Order System}
First a linear open-loop state space system is defined according to the measurement of a step on the robot slide, showing a time constant $\tau=110$\,ms, giving the closed-loop system in \autoref{eq:1storder_1D_ss}:
\begin{equation}
\dot{x} = Ax+Bu = Ax+B(\bar{N}x_{ref}-Kx) = %-\tau^{-1}x+\tau^{-1}u=
-\tau^{-1}x+\tau^{-1}(\bar{N}x_{ref}-Kx)
\label{eq:1D_1storder}
\end{equation} 
which can be recast as the augmented state-space system
\begin{equation}
\dot{x}=
\dot{\begin{bmatrix}
	x_1\\x_{ref}
	\end{bmatrix}} =
\begin{bmatrix}
A-BK&B\bar{N}\\0&0
\end{bmatrix}
\begin{bmatrix}
x_1\\x_{ref}
\end{bmatrix}
= f_{cl}(x)
\end{equation}
with no dynamics on the reference. A controller $K$ is found according to pole placement as described in \autoref{sec:K_Nbar_1D_1storder}, i.e. for the first order 1D system in \autoref{eq:1D_1storder} a controller that is 10 times faster than the system will be $K=9$. Similarly the system scaling factor $\bar{N}$ is determined according to the method described in \autoref{sec:K_Nbar_1D_1storder} as $\bar{N}=10$.
The independent state variables are defined and the SOS program initialized as described in \autoref{sec:app_sostools_barrier_search}.

Now the $g_j$ functions in \autoref{eq:barrier_constraints_putinar} are constructed according to \autoref{fig:safe:overview}:
\begin{itemize}
%\itemsep-1.3mm
\itemsep-0.7mm
\item The set $\mathcal{X}$ is defined by constructing a (second order polynomial) function $g_{1}(x_1)\geq 0 \in [\Lambda_{\text{lim}-},\Lambda_{\text{lim}+}]= [-0.1,0.1]$\,m, delimiting the region of possible robot tool positions, and another function $g_2(x_{ref})\geq 0 \in [\Lambda_{h-}+\Delta,\Lambda_{h+}-\Delta]=[-0.1+\Delta,0.05-\Delta]$\,m, delimiting the region of allowable reference positions 
\end{itemize}

.\newline 

Now declare the state space variables as \verb|syms| or \verb|pvar|, initialize the SOS program with the system states and write the closed-loop system equation $f_{cl}$ with the symbolic state vector.

For the system defined, make a function $g$ that is positive in the region that will be defined as $\mathcal{X}$ i.e. has its zero level set at the desired border of the region. Make an SOS variable $q$ from a monomial vector in the state variables of appropriate degree (preferably as small as possible to keep the complexity of the problem as low as possible). Finally declare $B$ as an SOS polynomial with a monomial vector of appropriate degree and set up the SOS inequality according to Putinar's Positivstellensatz in



Similarly, the region $\mathcal{X}_u$ can be defined as the surface of the heart, i.e. for the 1D case the robot tool is in the unsafe region if it is below the surface of the heart (if $x_2>x_1$), using the inequality on $B(x)$ in \autoref{cer2_app}.
%Additionally a physical amplitude constraint for the heart of 2\,cm is included in the definition of $\mathcal{X}_u$.

And finally the region $\mathcal{X}_0$ is defined as an area above the sine amplitude of the heart, in this case within slide positions 5-7\,cm, using the SOS inequality \autoref{cer1_app}. 
\textcolor{red}{Careful, the amplitude of the sine is not anywhere in the equations, maybe there should be a constraint on the maximum value of x2 or x3? where?}

\section{Safety Verification of Second Order System}
2nd order model in \autoref{eq:2ndorder_1D_ss}, determine K and Nbar in \autoref{sec:K_Nbar_1D_2ndorder} \textcolor{red}{Not written yet}