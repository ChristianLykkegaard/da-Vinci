First the open-loop state space system is defined, and a controller is found according to pole placement or another preferred method. Now declare the state space variables as \verb|syms| or \verb|pvar|, initialize the SOS program with the system states and write the closed-loop system equation $f_{cl}$ with the symbolic state vector.

For the system defined, make a function $g$ that is positive in the region that will be defined as $\mathcal{X}$ i.e. has its zero level set at the desired border of the region. Make an SOS variable $q$ from a monomial vector in the state variables of appropriate degree (preferably as small as possible to keep the complexity of the problem as low as possible). Finally declare $B$ as an SOS polynomial with a monomial vector of appropriate degree and set up the SOS inequality according to Putinar's Positivstellensatz in

\begin{lstlisting}[language=matlab]
% Define state-space system with x1 = robot position, x2 = heart position, x3 = heart velocity
tau = 1; % time constant for the robot slide
a = -1/tau;
Hz = 1; % frequency of the heart beat
b = 2*Hz;
A = [a   0  0; % robot tool position
0   0  b; % heart surface point position: sine wave
0 -b   0]; % heart surface point velocity
B = [1/tau 0 0]';
C = [1 0 0]; % the output is the robot tool position
poles = -10; % controller speed
k = place(A(1,1),B(1,1),poles);
K = [k 0 0]; % controller for robot tool position

% Reference signal for robot position: the heart position
Nbar = -inv(C*inv(A-B*K)*B);
ref = x2 * Nbar;
\end{lstlisting}

Similarly, the region $\mathcal{X}_u$ can be defined as the surface of the heart, i.e. for the 1D case the robot tool is in the unsafe region if it is below the surface of the heart (if $x_2>x_1$), using the inequality on $B(x)$ in \autoref{cer2_app}.
%Additionally a physical amplitude constraint for the heart of 2\,cm is included in the definition of $\mathcal{X}_u$.

And finally the region $\mathcal{X}_0$ is defined as an area above the sine amplitude of the heart, in this case within slide positions 5-7\,cm, using the SOS inequality \autoref{cer1_app}. 
\textcolor{red}{Careful, the amplitude of the sine is not anywhere in the equations, maybe there should be a constraint on the maximum value of x2 or x3? where?}