\section*{General Nomenclature Remarks}
\vspace{0.1cm}
\begin{itemize}
\item A dot above symbols indicate exclusively the time derivative, e.g. $\dot{x} = \dfrac{d}{dt}x(t)$
\item Well, maybe there is more
\item Or even more..
\end{itemize}

\textcolor{white}{\gls{analytic_func} \gls{rational_func} \gls{proper_func} \gls{injective_func} \gls{surjective_func} \gls{bijective_func} \gls{lipschitz} \gls{compact_space} \gls{hurwitz} \gls{dimension} \gls{order}}


\textcolor{red}{something simlar to this from \citep{bib:barrier_prajna}}
Notations: Most of the notations are standard. We denote
the set of real numbers by and the Euclidean n-space by $\mathbb{R}$.
The trace of an nxn matrix M, i.e., the sum of its diagonal elements,
is denoted by Tr(M). By f:X->Y we mean a function
mapping X subset Rn to Y subset Rm. We denote the spaces of
k-times continuously differentiable functions mapping
to Rm by Ck(X,Rm), and when m=1 we will write Ck(X).
Correspondingly, the spaces of continuous functions on are
denoted by C(X,Rn) and C(X). For a differentiable function
F:Rn->R, we use dF/dx(x) to denote the row vector
of partial derivatives of F with respect to x1,..xn. The Hessian
of a twice-differentiable function F:Rn->R is denoted
by d2F/dx2(x).