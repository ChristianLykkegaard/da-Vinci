\section*{General Notation Remarks}
\vspace{0.1cm}

Vectors are written i bold upright lowercase letters e.g. \textbf{x}, and vector entries are written as the same italic lowercase letter with a subscript generally denoting its entry e.g. $x_1$. The composition of the entries will be clear from the context.
Matrices are written in bold upright uppercase letters e.g. \textbf{A}, its transpose is denoted by \textbf{A}$^T$ and its inverse  is denoted by \textbf{A}$^{-1}$.

The $n$-dimensional real Euclidian space is denoted by $\mathbb{R}^n$ and subsets of the real space are written in calligraphic letters e.g. $\mathcal{X}\subseteq \mathbb{R}^n$. 
A function is written in italic letters followed by the variable(s) it is a function of e.g. $f(\mathbf{x})$. A differentiable function defined on $\mathbb{R}^n$ is denoted by $f\in C^1(\mathbb{R})$.

The time derivative of a variable is indicated by a dot above the symbol e.g. $\dot{\mathbf{x}} = d\mathbf{x}(t)/dt$. In general the notation $(t)$ denoting a function of time is implicit for the state vector \textbf{x}, and is only included to emphasize the time dependency. For functions of the state e.g. $q(\mathbf{x})$  the notation $(\mathbf{x})$ is left out when the dependency is clear from the context. The derivative notation $dB(\mathbf{x})/d\mathbf{x}$ implies the row vector of partial derivatives of $B$ with respect to $x_1,...,x_n$. 

The function $B(\mathbf{x})$ should not be confused with the matrix \textbf{B}.









\textcolor{white}{%\gls{analytic_func} \gls{rational_func} \gls{proper_func} \gls{lipschitz} \gls{dimension} \gls{hurwitz} 
	\gls{injective_func} \gls{surjective_func} \gls{bijective_func}  \gls{compact_space}  \gls{extrinsic} \gls{intrinsic} \gls{tcp} \gls{Ts} \gls{y} \gls{radius_vec} \gls{center_vec}}