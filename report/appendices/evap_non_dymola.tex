This appendix serves as documentation that the mass and energy balance can be written with the following two state variables solely:
\begin{itemize}
\item Evaporator vapor volume, $V_\text{E,V}$
\item The pressure within the Evaporator, $p_E$
\end{itemize}
The derivation throughout this appendix assumes that the following variables are known from elsewhere:
\begin{flalign}
\lbrace m_9, h_9, T_9,m_{10},h_{10},T_{10}, Q_E \rbrace \kk \text{The variables at \textbf{9}  + $Q_E$ are inputs. The rest are assumed known.} \nonumber
\end{flalign}
This appendix serves furthermore as a production of evidence that the rest of the components i.e. the Absorber, Generator, Condensor, solution pump and valves can be derived in a similar way. Due to the extensive amount of derivative terms, and the repeated work, these are not performed here.
\section{Mass Balance}
The mass balance is first stated from \autoref{eq:nemt1} as:
\pp
\dfrac{dM_\text{E,tot}}{dt} = m_{10} - m_9 
\end{flalign}
Splitting the total mass up into liquid and vapor fractions:
\pp
\dfrac{d(V_\text{E,L}\,\rho_\text{E,L} + V_\text{E,V}\,\rho_\text{E,V})}{dt} = \dfrac{d(V_\text{E,L}\,\rho_\text{E,L})}{dt}+\dfrac{d( V_\text{E,V} \, \rho_\text{E,V} )}{dt} = m_{10} - m_{9} \label{eq:evap_intro}
\end{flalign}
Defining the liquid volume as a function of the total volume and vapor volume to obtain:
\pp
V_\text{E,L} = V_\text{E,tot}-V_\text{E,V} \label{eq:volume_app}
\end{flalign}
Inserting this in \autoref{eq:evap_intro}:
\pp
&\dfrac{d( V_\text{E,tot} - V_V)\,\rho_\text{E,L} }{dt} + \dfrac{ (V_\text{E,V}\, \rho_\text{E,V}) }{dt} = m_{10} - m_9 \kk \Leftrightarrow \nonumber \\
& \dfrac{d(  V_\text{E,tot}\,\rho_\text{E,L} - V_\text{E,V}\,\rho_\text{E,L})}{dt}  + \dfrac{ (V_\text{E,V}\, \rho_\text{E,V}) }{dt} = m_{10} - m_9 \kk \Leftrightarrow \nonumber \\
&\dfrac{d(V_\text{E,tot}\, \rho_\text{E,L} )}{dt} - \dfrac{d( V_\text{E,V}\,\rho_\text{E,L} )}{dt} + \dfrac{ (V_\text{E,V}\, \rho_\text{E,V}) }{dt} = m_{10} - m_9 \label{eq:E:evap2}
\end{flalign}
By applying the product rule of derivative terms ($(f\cdot g)'=f'\cdot g+f\cdot g'$)) and utilizing that $V_\text{E,tot}$ is a constant, \autoref{eq:E:evap2} can be written as:
\pp
V_\text{E,tot}\,\dfrac{d(\rho_\text{E,L})}{dt} - \left(  \dfrac{d(V_\text{E,V})}{dt}\,\rho_\text{E,L} + V_\text{E,V}\, \dfrac{d(\rho_\text{E,L})}{dt}   \right) + \dfrac{d(V_\text{E,V})}{dt}\, \rho_\text{E,V} + V_\text{E,V}\,\dfrac{d( \rho_\text{E,V} )}{dt} = m_{10} - m_9 \label{akgogg}
\end{flalign}
The liquid and vapor density are functions of both pressure and temperature for which it should be derived further as:
\pp
&\dfrac{\partial (\rho_\text{E,L})}{\partial t} = \dfrac{\partial (\rho_\text{E,L})}{\partial p_E}\cdot \dfrac{\partial (p_E)}{\partial t} + \dfrac{\partial (\rho_\text{E,L})}{\partial T_E} \cdot \dfrac{\partial (T_E)}{\partial t}
\kk { \scriptscriptstyle \text{assume liquid is in saturation}   \above0pt \longrightarrow } \\
& \dfrac{\partial (\rho_\text{E,L})}{\partial t} = \dfrac{d ( \rho_\text{E,L} )}{dt} = \dfrac{d ( \rho_\text{E,L} )}{dp_E}\cdot \dfrac{dp_E}{dt} \kk \text{This applies for the vapor as well} \label{eq:tiln}
\end{flalign}
As it is fair to assume that the evaporator liquid and vapor volume are in saturation. For that reason,  \autoref{akgogg} can be written further as (by use of \autoref{eq:tiln}):
\pp
&V_\text{E,tot}\,\dfrac{d ( \rho_\text{E,L} )}{dp_E}\cdot \dfrac{dp_E}{dt} - \left(  \dfrac{d(V_\text{E,V})}{dt}\,\rho_\text{E,L} + V_\text{E,V}\, \dfrac{d ( \rho_\text{E,L} )}{dp_E}\cdot \dfrac{dp_E}{dt}   \right) + \dfrac{d(V_\text{E,V})}{dt}\, \rho_\text{E,V} + V_\text{E,V}\,\dfrac{d ( \rho_\text{E,V} )}{dp_E}\cdot \dfrac{dp_E}{dt} \nonumber \\ &= m_{10} - m_9 \label{eq:mass_evap_comp}
\end{flalign}
Thus \autoref{eq:mass_evap_comp} describes the mass balance with the two state variables $p_E$ and $V_\text{E,V}$. All derivative terms with respect to the pressure can be found as look up tables.
\section{Energy Balance}
The energy balance can in the same way be described solely by the same two state variables. The energy balance is recapitulated from \autoref{eq:nemt1}. The part $\rho \cdot V$ is assumed to be insignificant such that the internal energy reduces to the enthalpy $(H = U + \text{\st{$V\cdot p$}}) \, \Rightarrow \, H = U$. This is due to the huge physical size of the system. This might not be the case in smaller systems.
\pp
\dfrac{d(U)}{dt} \approx
\dfrac{d(H)}{dt} = m_9\,h_9 - m_{10}\,h_{10} + Q_E
\end{flalign}
$h_{10}$ is replaced by $c_V\, T_E$ (specific heat capacity for vapor and Evaporator temperature) and similar to $h_9$:
\pp
\dfrac{d(H)}{dt} = m_9\,\,c_L\, T_9  - m_{10}\,c_V\, T_E + Q_E \label{eq:dh_evap_app}
\end{flalign}
The enthalpy $H$ is replaces by:
\pp
&H = H_\text{E,L} + H_\text{E,V} = M_\text{E,L}\,h_\text{E,L}(p_E) + M_\text{E,V}\,h_\text{E,V}(p_E) \kk \text{note: $H = M\cdot h$ and that the}\label{eq:H_evap_app}\\ &\text{specific enthalpy is a function of the pressure only as both vapor and liquid are in saturation.}\nonumber
\end{flalign}
Inserting \autoref{eq:H_evap_app} into \autoref{eq:dh_evap_app}:
\pp
\dfrac{d(M_\text{E,L}\,h_\text{E,L}(p_E) + M_\text{E,V}\,h_\text{E,V}(p_E))}{dt} = m_9\,\,c_L\, T_9  - m_{10}\,c_V\, T_E + Q_E 
\end{flalign}
Splitting the derivative up into two terms:
\pp
\dfrac{d(M_\text{E,L}\,h_\text{E,L}(p_E))}{dt} + \dfrac{d(M_\text{E,V}\,h_\text{E,V}(p_E))}{dt} = m_9\,\,c_L\, T_9  - m_{10}\,c_V\, T_E + Q_E 
\end{flalign}
Substituting $M_n=V_n\,\rho_n$:
\pp
\dfrac{d(V_\text{E,L}\,\rho_\text{E,L}\,h_\text{E,L}(p_E))}{dt} + \dfrac{d(V_\text{E,V}\,\rho_\text{E,V}\,h_\text{E,V}(p_E))}{dt} = m_9\,\,c_L\, T_9  - m_{10}\,c_V\, T_E + Q_E 
\end{flalign}
Applying same trick as in \autoref{eq:volume_app}:
\pp
&\dfrac{d((V_\text{E,tot}-V_\text{E,V} )\,\rho_\text{E,L}\,h_\text{E,L}(p_E))}{dt} + \dfrac{d(V_\text{E,V}\,\rho_\text{E,V}\,h_\text{E,V}(p_E))}{dt} = m_9\,\,c_L\, T_9  - m_{10}\,c_V\, T_E + Q_E \kk \Leftrightarrow \nonumber\\
& \dfrac{d((V_\text{E,tot}\,\rho_\text{E,L}\,h_\text{E,L}(p_E) - V_\text{E,V}\,\rho_\text{E,L}\,h_\text{E,L}(p_E) )}{dt} + \dfrac{d(V_\text{E,V}\,\rho_\text{E,V}\,h_\text{E,V}(p_E))}{dt} = m_9\,\,c_L\, T_9  - m_{10}\,c_V\, T_E + Q_E  \kk \Leftrightarrow \nonumber \\
&\underbrace{ \dfrac{d((V_\text{E,tot}\,\rho_\text{E,L}\,h_\text{E,L}(p_E)}{dt} }_1 - \underbrace{\dfrac{d(V_\text{E,V}\,\rho_\text{E,L}\,h_\text{E,L}(p_E) )}{dt} }_2 +  \underbrace{ \dfrac{d(V_\text{E,V}\,\rho_\text{E,V}\,h_\text{E,V}(p_E))}{dt} }_3 = m_9\,\,c_L\, T_9  - m_{10}\,c_V\, T_E + Q_E \nonumber
\end{flalign}
By assuming that the density for liquid, $\rho_\text{E,L}$, is constant, by applying the product rule of derivative terms ($(f\cdot g)'=f'\cdot g+f\cdot g'$)), the chain rule ($\,\,\tfrac{d\, z(y)}{dx} = \tfrac{d z}{dy}\cdot \tfrac{d y}{dx} \,\,$) and by utilizing that $V_\text{E,tot}$ is a constant (under braces are added to help the reader keep track of the derivative terms):
\pp
& \underbrace{ V_\text{E,tot}\,\rho_\text{E,L}\, \dfrac{d(h_\text{E,L})}{dp_E}\cdot \dfrac{dp_E}{dt} }_1   -  \underbrace{ 
\rho_\text{E,L} \left( \dfrac{d(V_\text{E,V})}{dt}\,h_\text{E,L}(p_E) +  V_\text{E,V}\,\dfrac{d(h_\text{E,L}(p_E))}{dt} \right)}_2  +     \nonumber \\
&\underbrace{ h_\text{E,V}(P_E)\,\rho_\text{E,V}(p_E)\,\dfrac{d(V_\text{E,V})}{dt} + 
 h_\text{E,V}(p_E)\, V_\text{E,V}\,\dfrac{d(\rho_\text{E,V})}{dt} + V_\text{E,V}\,\rho_\text{E,V}(p_E)\, \dfrac{d(h_\text{E,V}(p_E) )}{dt} }_3  \nonumber \\& =  m_9\,\,c_L\, T_9  - m_{10}\,c_V\, T_E + Q_E  \nonumber
\end{flalign}
Applying the chain rule to all derivative terms of $h_n$ w.r.t. time as $h$ as a state variable is undesired and applying the chain rule to derivative terms of $\rho_\text{E,V}$ w.r.t. time as $\rho$ as a state variable is undesired:
\pp
& \underbrace{ V_\text{E,tot}\,\rho_\text{E,L}\, \dfrac{d(h_\text{E,L})}{dp_E}\cdot \dfrac{d(p_E)}{dt} }_1   -  \underbrace{ 
\rho_\text{E,L} \left( \dfrac{d(V_\text{E,V})}{dt}\,h_\text{E,L}(p_E) +  V_\text{E,V}\,   \dfrac{d(h_\text{E,L})}{dP_E}\cdot \dfrac{d(p_E)}{dt} \right)}_2  +  \nonumber \\
&\underbrace{ h_\text{E,V}(p_E)\, \rho_\text{E,V}(p_E)  \,     \,\dfrac{d(V_\text{E,V})}{dt} + 
 h_\text{E,V}(p_E)\, V_\text{E,V}\,  \dfrac{d(\rho_\text{E,V})}{dp_E}\cdot \dfrac{d(p_E)}{dt}   + V_\text{E,V}\,\rho_\text{E,V}(p_E)\,  \dfrac{d(h_\text{E,V})}{dp_E}\cdot \dfrac{d(p_E)}{dt}     }_3  \nonumber\\
 &=  m_9\,\,c_L\, T_9  - m_{10}\,c_V\, T_E + Q_E  \label{eq:energy:final}
\end{flalign}
\section{State Space Representation}
The two final equations from \autoref{eq:mass_evap_comp} and  \autoref{eq:energy:final} can be summarized as the below written ones (slightly rearranged), where all constants are given shorter acronyms for a temporary enhanced overview. The notation of derivative terms is switched to the dot-notation. Mass balance:
\pp
& \underbrace{ V_\text{E,tot}\,\dfrac{d ( \rho_\text{E,L} )}{dp_E}}_a\cdot \dfrac{dp_E}{dt} - \underbrace{\rho_\text{E,L}}_b\, \dfrac{d(V_\text{E,V})}{dt} - \underbrace{\dfrac{d ( \rho_\text{E,V} )}{dp_E}}_c\,V_\text{E,V}\, \dfrac{dp_E}{dt}   +  \underbrace{\rho_\text{E,V}(p_E)}_d\,\dfrac{d(V_\text{E,V})}{dt} + \underbrace{ \dfrac{d ( \rho_\text{E,V} )}{dp_E}}_e\,V_\text{E,V}\cdot \dfrac{dp_E}{dt} \nonumber \\ &= \underbrace{ m_{10} - m_9 }_f \kk \Leftrightarrow \nonumber\\
&a\,\dot{p}_E - b\, \dot{V}_\text{E,V} - c\, V_\text{E,V}\, \dot{p}_E +  d \, \dot{V}_\text{E,V} + e\,V_\text{E,V}\, \dot{p}_E = f \kk \Leftrightarrow \nonumber \\
& (a-c\,V_\text{E,V}+e\,V_\text{E,V})\,\dot{p}_E + (-b+d)\,\dot{V}_\text{E,V} = f \kk \Leftrightarrow\nonumber\\
&(a-(c+e)\,V_\text{E,V})\,\dot{p}_E+(d-b)\,\dot{V}_\text{E,V} = f \label{eq:evap_mass_app_ss}
\end{flalign}
The energy balance is converted in a similar way:
\pp
&\underbrace{ V_\text{E,tot}\,\rho_\text{E,L}\, \dfrac{d(h_\text{E,L})}{dp_E}}_\alpha\cdot \dfrac{d(p_E)}{dt}  -  \underbrace{
\rho_\text{E,L}\,h_\text{E,L}(p_E)}_\beta\,  \dfrac{d(V_\text{E,V})}{dt} - \underbrace{ \rho_\text{E,L} \,   \dfrac{d(h_\text{E,L})}{dP_E}}_\gamma  \,  V_\text{E,V} \, \dfrac{d(p_E)}{dt}   +  \nonumber \\
& \underbrace{ h_\text{E,V}(p_E)\, \rho_\text{E,V}(p_E)}_\sigma     \, \dfrac{d(V_\text{E,V})}{dt} + 
\underbrace{ h_\text{E,V}(p_E)\,  \dfrac{d(\rho_\text{E,V})}{dp_E}}_\delta  \, V_\text{E,V} \, \dfrac{d(p_E)}{dt}   + \underbrace{ \rho_\text{E,V}(p_E)\,  \dfrac{d(h_\text{E,V})}{dp_E}}_\epsilon \,V_\text{E,V}\,\dfrac{d(p_E)}{dt}      \nonumber\\
&= \underbrace{ m_9\,\,c_L\, T_9  - m_{10}\,c_V\, T_E + Q_E}_\zeta \kk \Leftrightarrow \nonumber \\
&\alpha\, \dot{p}_E - \beta\,\dot{V}_\text{E,V} - \gamma\, V_\text{E,V}\, \dot{p}_E + \sigma\, \dot{V}_\text{E,V} + \delta\,V_\text{E,V}\, \dot{p}_E + \epsilon\,V_\text{E,V}\, \dot{p}_E = \zeta \kk \Leftrightarrow \nonumber \\
&(\alpha - \gamma\,V_\text{E,V}+\delta\,V_\text{E,V} + \epsilon\,V_\text{E,V})\,\dot{p}_E + (-\beta+\sigma)\,\dot{V}_\text{E,V} = \zeta \kk \Leftrightarrow \nonumber \\
&(  \alpha + (\delta + \epsilon - \gamma) V_\text{E,V} )\,\dot{p}_E + (\sigma - \beta)\,\dot{V}_\text{E,V} = \zeta \label{eq:evap_energy_app_ss}
\end{flalign}
Hence \autoref{eq:evap_mass_app_ss} and \autoref{eq:evap_energy_app_ss} can be outlined together into a descriptor form:
\pp
&(a-(c+e)\,V_\text{E,V})\,\dot{p}_E+(d-b)\,\dot{V}_\text{E,V} = f \\
&(  \alpha + (\delta + \epsilon - \gamma) V_\text{E,V} )\,\dot{p}_E + (\sigma - \beta)\,\dot{V}_\text{E,V} = \zeta \nonumber
\end{flalign}
Rewriting to a matrix-vector multiplication:
\pp
\begin{bmatrix}
(a-(c+e)\,V_\text{E,V}) &  (d-b)\\
(  \alpha + (\delta + \epsilon - \gamma) V_\text{E,V} ) &  (\sigma - \beta)
\end{bmatrix} \, 
\dot{\begin{bmatrix}
p_E \\ 
V_\text{E,V}
\end{bmatrix}} =
\begin{bmatrix}
f \\
\zeta
\end{bmatrix}
\end{flalign}
Isolating the state vector:
\pp
\dot{\begin{bmatrix}
p_E \\ 
V_\text{E,V}
\end{bmatrix}} = 
\begin{bmatrix}
(a-(c+e)\,V_\text{E,V}) &  (d-b)\\
(  \alpha + (\delta + \epsilon - \gamma) V_\text{E,V} ) &  (\sigma - \beta)
\end{bmatrix}^{-1}
\begin{bmatrix}
f \\ \zeta
\end{bmatrix}
\end{flalign}
The state variables are now expressed as a function of inputs and states. A linearization could convert them to the usual state space form $\dot{x} = A\,x+B\,u$. This is not performed here as the principle already is proven at this point in time.
