\glsreset{clf}\glsreset{cbf}
Based on \citep{bib:artstein}, who founded \glspl{clf}, a \gls{cbf} can be created \citep{bib:org_control}. 
\begin{exa}[Control Barrier Function]
Given a system $\dot{x}=f(x)+g(x)u$, a \gls{cbf} exists if the below constraints are fulfilled:
\begin{subequations}
\begin{flalign}
x\in \mathcal{X}_u \hspace{0.3cm} &\Rightarrow \hspace{0.3cm} B(x) > 0  \label{req1} \\
L_gB(x) = 0 \hspace{0.3cm} &\Rightarrow \hspace{0.3cm} L_fB(x) < 0 \label{req2} \\
\{ x \in \mathcal{X} | B(x) \leq 0 \} &\neq \emptyset \label{req3}
\end{flalign}
\end{subequations}
\vspace{-0.6cm}
\begin{tabular}{r l l} 
where  & & \\
$B(x)$ & is a control barrier function & [$\cdot$] \\ 
$L_fB(x)$ & is the Lie derivative of $B(x)$ along the vector field  $f(x)$, i.e. $\frac{\partial B(x)}{\partial x}f(x)$ & [$\cdot$] \\ 
$L_gB(x)$ & is the Lie derivative of $B(x)$ along the vector field  $g(x)$, i.e. $\frac{\partial B(x)}{\partial x}g(x)$ & [$\cdot$] 
\end{tabular}
\vspace*{-0.2cm}
\end{exa}
Taking a look at \autoref{req1} it states essentially the same as \autoref{cer2}, i.e. the unsafe area exist whenever $B(x)>0$. This makes it possible to design an unsafe region. \Autoref{req2} put forth the requirement that the gradient along the vector field $f(x)$ must point away from the barrier extremities whenever the input is with no significance (except for the critical point). \Autoref{req3} simply states that the safe area must contain some states as control otherwise is impossible.

This chapter intends to implement and analyse a controller ensuring safety if the demands from \autoref{req1}, \ref{req2} and \ref{req3} are obeyed. This shall first be tested on the slide movement on the Da Vinci surgical robot as it composes a prismatic joint and a 1:1 mapping from slide $joint\_angle$ to 1D position. Hence any inverse kinematic solver can be bypassed in the early phase of this project which is an important simplification to eliminate initial complications.
