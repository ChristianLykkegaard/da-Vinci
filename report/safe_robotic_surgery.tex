\section{Safety Precautions for Automated Surgery Control}
The system considered will be the general non-linear system on the form:
\begin{flalign*}
\dot{x} = f(x) + g(x)\,u + h(x)\,d
\end{flalign*}
\vspace{-0.7cm}
\begin{longtable}{p{.9\textwidth} p{.1\textwidth} p{.1\textwidth}} 
where & & \\
\gls{x} is the state and restricted to $x \in \mathbb{R}^n$ where  \gls{n} is the number of states &[$\cdot$]& \\
\gls{u} is the control input and restricted to $u \in \mathbb{R}^m$ where \gls{m} is the number of inputs& [$\cdot$]& \\
\gls{d} is the disturbance input and restricted to $d \in D \subseteq \mathbb{R}^p$ where \gls{p} is the number of disturbances & [$\cdot$]& \\
\gls{f} is a potential non-linear function, $f:\mathbb{R}^n \rightarrow \mathbb{R}^n$ & [$\cdot$]& \\
\gls{g} is a potential non-linear function, $g:\mathbb{R}^n \rightarrow \mathbb{R}^{n \times m}$ & [$\cdot$]& \\
\gls{h} is a potential non-linear function, $h:\mathbb{R}^n \rightarrow \mathbb{R}^{n \times p}$ & [$\cdot$]& 
\end{longtable}

The definition of safety will follow the definition described in \citep{bib:safety}, i.e.:
\begin{exa}
A closed loop control system, $\Gamma_\text{cl} = (f_\text{cl},h,X,X_0.X_u,D)$, is unsafe if there exist a time $t \in [0,$\gls{T}$]$ such that the trajectory $\phi_{X_0}^{\bar{d}} : [0,T]$ satisfy: 
	\begin{flalign}
		\left( \phi_{X_0}^{\bar{d}}([0,t]) \cap X_u \right) \neq \emptyset \kk \wedge \kk 
		\phi_{X_0}^{\bar{d}}([0,t]) \subseteq X
	\label{eq:defsafety}
	\end{flalign}
The closed loop system $\Gamma_\text{cl}$ is safe if there are no unsafe trajectories.
\vspace{-0.2cm}
\begin{longtable}{p{.9\textwidth} p{.1\textwidth} p{.1\textwidth}} 
Where  & & \\
\gls{fcl} is a potential non-linear function with the closed loop characteristic:\\ \kk $f_\text{cl}: x \mapsto f(x)+g(x)k(x)$ where \gls{k} is the feedback gain with the map $k: \mathbb{R}^n \rightarrow \mathbb{R}^m$ & [$\cdot$] &  \\
\gls{X} is the set of all allowed states & [$\cdot$] &  \\
\gls{X0} is the set of all allowed initial states & [$\cdot$] &  \\
\gls{Xu} is the set of all unsafe states & [$\cdot$] &  \\
\gls{phi} is the set of all allowed initial conditions with the bounded disturbance input \gls{dbar} & [$\cdot$]
\end{longtable}
A graphical interpretation can be deduced from \autoref{eq:defsafety} and found in \autoref{fig:defsafety}.
\begin{figure}[H]
	\center
		\includegraphics[width=0.75\textwidth]{safety.pdf}	
	\caption{Graphically interpretation of \autoref{eq:defsafety} in the state space. The blue trajectory is unsafe while the green trajectory is safe.}
	\label{fig:defsafety}
\end{figure}
\label{def_safety}
\end{exa}

\textbf{Beskriv forskellige m\aa der at h\aa ndtere constraint, samt beskriv deres fordele og ulemper!}

- \gls{mpc} \\
- \gls{clf} combined wiht \gls{cbf} (tabilization property of CLF
with the safety aspect from the CBF) =  \\
- reference governor?\\

\gls{lie}

\section{Control Barrier Functions}
The control of the safety problem can be carried out by inspiration of ordinary CLFs (Control Lyapunov Functions) \citep{bib:org_control}