Sequence of tasks for static trajecory tracking
\vspace{-3mm}
\begin{itemize}
	\itemsep-1.3mm
	\item User inputs desired final (3D) setpoint for the end-effector (in inertial frame)
	\item EITHER: The ROS OMPL computes a trajectory avoiding defined obstacles \\
	OR: $\qquad$We design an algorithm that does the path planning, based on a Lyapunov function in combination with a barrier certificate
	\item From the computed trajectory, a sequence of appropriate setpoints for the low-level controllers are generated
	\item ROS uses its inverse kinematics solvers to compute the desired state for each of the generated setpoints
	\item The low-level controllers implemented in the hardware handles the control of the setpoint tracking
\end{itemize}

Sequence of tasks for dynamic surface point tracking
\vspace{-3mm}
\begin{itemize}
	\itemsep-1.3mm
	\item User inputs desired relative position (distance and orientation) of end-effector frame with respect to the heart (surface point) frame, see \autoref{app:kinematic_model_robot} and \ref{app:dynamic_model_heart}
	\item Configuration of heart frame is computed relative to time (determine $t_0$), and from this the absolute end-effector frame is computed (real-time) - also include barrier certificates to avoid heart penetration?
	\item ROS uses its inverse kinematics solvers to compute the desired state for this configuration (real-time)
	\item The low-level controllers implemented in the hardware handles the control of the configuration tracking (real-time)
\end{itemize}

It is desired at the end of the project to perform a trajectory following task with the robot arm (not only simulations or visualization through the ROS tool Rviz), and for that a tissue phantom must be procured, which allows for multiple tests with cutting tools.

\section{Safety Precautions for Automated Surgery Control}\label{sec:safety-def}
A crucial matter when dealing with those topics within robotic surgeries feature necessary conditions to guarantee the patient safety and to avert patient trauma \citep{bib:safety}.




The system considered will be the general non-linear system on the form:
\begin{flalign*}
\dot{x} = f(x) + g(x)\,u + h(x)\,d
\end{flalign*}
\vspace{-0.7cm}
\begin{longtable}{p{.9\textwidth} p{.1\textwidth} p{.1\textwidth}} 
where & & \\
\gls{x} is the state and restricted to $x \in \mathbb{R}^n$ where  \gls{n} is the number of states &[$\cdot$]& \\
\gls{u} is the control input and restricted to $u \in \mathbb{R}^m$ where \gls{m} is the number of inputs& [$\cdot$]& \\
\gls{d} is the disturbance input and restricted to $d \in D \subseteq \mathbb{R}^p$ where \gls{p} is the number of disturbances & [$\cdot$]& \\
\gls{f} is a potential non-linear function, $f:\mathbb{R}^n \rightarrow \mathbb{R}^n$ & [$\cdot$]& \\
\gls{g} is a potential non-linear function, $g:\mathbb{R}^n \rightarrow \mathbb{R}^{n \times m}$ & [$\cdot$]& \\
\gls{h} is a potential non-linear function, $h:\mathbb{R}^n \rightarrow \mathbb{R}^{n \times p}$ & [$\cdot$]& 
\end{longtable}

The definition of safety will follow the definition described in \citep{bib:safety}, i.e.:
\begin{exa}
A closed loop control system, $\Gamma_\text{cl} = (f_\text{cl},h,X,X_0.X_u,D)$, is unsafe if there exist a time $t \in [0,$\gls{T}$]$ such that the trajectory $\phi_{X_0}^{\bar{d}} : [0,T]$ satisfy: 
	\begin{flalign}
		\left( \phi_{X_0}^{\bar{d}}([0,t]) \cap X_u \right) \neq \emptyset \kk \wedge \kk 
		\phi_{X_0}^{\bar{d}}([0,t]) \subseteq X
	\label{eq:defsafety}
	\end{flalign}
The closed loop system $\Gamma_\text{cl}$ is safe if there are no unsafe trajectories.
\vspace{-0.2cm}
\begin{longtable}{p{.9\textwidth} p{.1\textwidth} p{.1\textwidth}} 
Where  & & \\
\gls{fcl} is a potential non-linear function with the closed loop characteristic:\\ \kk $f_\text{cl}: x \mapsto f(x)+g(x)k(x)$ where \gls{k} is the feedback gain with the map $k: \mathbb{R}^n \rightarrow \mathbb{R}^m$ & [$\cdot$] &  \\
\gls{X} is the set of all allowed states & [$\cdot$] &  \\
\gls{X0} is the set of all allowed initial states & [$\cdot$] &  \\
\gls{Xu} is the set of all unsafe states & [$\cdot$] &  \\
\gls{phi} is the set of all allowed initial conditions with the bounded disturbance input \gls{dbar} & [$\cdot$]
\end{longtable}
A graphical interpretation can be deduced from \autoref{eq:defsafety} and found in \autoref{fig:defsafety}.
\begin{figure}[H]
	\center
		\includegraphics[width=0.75\textwidth]{safety.pdf}	
	\caption{Graphically interpretation of \autoref{eq:defsafety} in the state space. The blue trajectory is unsafe while the green trajectory is safe.}
	\label{fig:defsafety}
\end{figure}
\label{def_safety}
\end{exa}

\textbf{Beskriv forskellige m\aa der at h\aa ndtere constraint, samt beskriv deres fordele og ulemper!}

- \gls{mpc} \\
- \gls{clf} combined wiht \gls{cbf} (stabilization property of CLF
with the safety aspect from the CBF)   \\
- reference governor?\\

\gls{lie}

If the system always moves very slowly, acceleration may be ignored and maybe dyamics could be ignored altogether, only leaving the evolution of kinematics in the function $f(x)$.

The design of a safe controller features the property that a supplied control signal ensures compliance of the definition described in \autoref{sec:safety-def}.

A one-dimensional simple case is analysed at first. The approach adopted constitute the movement of instrument slide, see \autoref{fig:lol}.

\begin{figure}[H]
\center
--- NICE FIGURE OF INSTRUMENT SLIDE ---
\caption{nice figure}
\label{fig:lol}
\end{figure}

A barrier certificate function can be constructed from the definitions, i.e.:
\begin{flalign}
& B(x) \leq 0 \kk  \forall \hspace{0.3cm} x \in \mathcal{X}_0  \label{cer1}\\
& B(x) > 0  \kk  \forall \hspace{0.3cm} x \in \mathcal{X}_u \label{cer2} \\
& L_f(B(x)) \leq 0 \kk  \forall \hspace{0.3cm} x \in \mathcal{X} \label{cer3}
\end{flalign}
\vspace{-0.8cm}
\begin{longtable}{p{.9\textwidth} p{.1\textwidth} p{.1\textwidth}} 
Where  & & \\
\gls{bar} is the barrier function & [$\cdot$] \\ 
\end{longtable}
\vspace*{-0.2cm}
Based on \citep{bib:artstein}, which founded \gls{clf}s, a \gls{cbf} can be created \citep{bib:org_control}. With a system $\dot{x}=f(x)+g(x)u$, a \gls{cbf} exist if the below constraints are fulfilled:
\begin{flalign}
& x\in \mathcal{X}_u \hspace{0.3cm} \Rightarrow \hspace{0.3cm} B(x) > 0  \label{req1} \\
& L_gB(x) = 0 \hspace{0.3cm} \Rightarrow \hspace{0.3cm} L_fB(x) < 0 \label{req2} \\
& \{ x \in \mathcal{X} | B(x) \leq 0 \} \neq \emptyset \label{req3}
\end{flalign}
\vspace{-0.8cm}
\begin{longtable}{p{.9\textwidth} p{.1\textwidth} p{.1\textwidth}} 
Where  & & \\
$L_f(x)$ is the Lie derivative of $B(x)$ along the vector field  $f(x)$, i.e. $\frac{\partial B(x)}{\partial x}f(x)$ & [$\cdot$] \\ 
$L_f(x)$ is the Lie derivative of $B(x)$ along the vector field  $g(x)$, i.e. $\frac{\partial B(x)}{\partial x}g(x)$ & [$\cdot$] 
\end{longtable}
\vspace*{-0.2cm}
Taking a look at \autoref{req1} it states essentially the same as \autoref{cer2}, i.e. the unsafe area exist whenever $B(x)>0$. This makes it possible to design a unsafe region. \Autoref{req2} put forth the requirement that the gradient along the vector field $f(x)$ must point away from the barrier extremities whenever the input is with no significance (except for the critical point). \Autoref{req3} simply states that the safe area must contain some states as control otherwise is impossible.
\section{1-Dimensional Controller Design for Safety}
\subsubsection{Modelling of Slide Movement}
The slide movement is visualized in \autoref{fig:slidefig}.
\begin{figure}[H]
\center
\includegraphics[scale=0.2]{slidemovefigure.pdf}
\caption{Slide movement which takes placed in the z-plane.}
\label{fig:slidefig}
\end{figure}

To obtain a model, a step response will be performed on the slide movement. The slide position can be measured by publishing the joint state angle in \gls{ros}. The experiment is described in further details in \autoref{app:meas}.

%\hspace{1cm }\texttt{rostopic echo joint\_states/position[6]} \hspace{0.2cm} {\color{blue}{\# Be sure to have the ROS environment correctly configured according to \autoref{app:ros}}}

The step response is plotted in \autoref{fig:stepresponseslide}.
\begin{figure}[H]
\center
\includegraphics[scale=0.5]{step_slide.eps}
\caption{Step response from 0\,mm to 5\,mm. Plot details and measurements can be found in \autoref{app:cd} as \texttt{matlab\_scripts/slide\_step/plot\_slide\_pos.m}}
\label{fig:stepresponseslide}
\end{figure}

It is clear that this could be well approximated with an underdamped second order model (complex roots), however for initial simplicity and because modelling is not a focus point in this thesis, merely a simple model of the slide movement is used (it shall later be approximated to a second order model). Therefore, with some good will, it is approximated to a linear first order system with a dominating time constant \gls{taus}: 
\begin{flalign*}
& Y(s) = \dfrac{1}{\tau_s s + 1}U(s) =  \dfrac{1/\tau_s}{s + 1/\tau_s}\,U(s) = (s+1/\tau_s)^{-1}\,1/\tau_s\,U(s) \kk  \overset{\overset{Y(s)=(C(sI-A)^{-1}B+D)U(s)}{\longrightarrow}}{\scriptsize \text{compare to obtain SS form}}  \\ 
& \dot{x} = \underbrace{-\tau_s^{-1}\,x}_{f_s(x)} + \underbrace{\tau_s^{-1}}_{g_s(x)} u
\end{flalign*}
Thus the system matrix $A$ and the input matrix $B$ can be seen easily. For the sake of generalization, they are named $f_s$ and $g_s$, i.e.:
\begin{flalign*}
f_s(x) = -\tau^{-1}x \kk \wedge \kk g_s(x) = \tau_s^{-1}
\end{flalign*}
A suitable time constant $\tau_s$ can be read from \autoref{fig:stepresponseslide} to:
\begin{flalign*}
\tau_s = 55\, \text{ms}
\end{flalign*} 
\subsubsection{Construction of Simple CBF}
To illustrate the usefulness of \gls{cbf}s, a palpable example hereof will be created with direct application to the Da Vinci robot. This example does not directly constitute application to a patient but favour the theory in a neat and comprehensible sense and secure a way to visually and physically verify the method.

A parabola is introduced as \gls{cbf}. A coordinate shift is performed such that the slide movement occurs along the $x$-axis instead of the $z$-axis. 
\begin{flalign*}
B(x) = ax^2+bx+c
\end{flalign*}
Defining the sets:
\begin{table}[H]
	\begin{tabularx}{\textwidth}{X X X X}
\rowcolor{HeaderBlue} 
$\mathcal{X}$ & $\mathcal{X}_u$ & $\bar{\mathcal{X}}_u$ & $\mathcal{X}_0$ \\
$x \in [-0.10:0.10]$ & $x \in [0.05:0.10] $ & $ x \in [-0.10:0.05] $ & $x \in [-0.10:0.05]$  \\
\end{tabularx}
\end{table}
These state intervals are indeed the immediate states where one would address their attention towards. However, the controller designed for safety does not necessarily have to be active before the slide position is within some distance $d_s$ (determined by a parameter $\epsilon$) of a hard boundary $\Psi_h$. This crossing line shall be denoted as a soft boundary $\Psi_s$. See ??? for an overview of these parameters. This does indeed suggest for a divided control law \citep{bib:org_control}. Consider the control law below:
\begin{flalign*}
u(x) = \sigma(x)k_0(x)+(1-\sigma(x))\tilde{u}(x) = \sigma(x)k_0(x)+(1-\sigma(x))(\bar{N} \cdot x_\text{ref}-Kx) 
\end{flalign*}
\vspace{-0.8cm}
\begin{longtable}{p{.9\textwidth} p{.1\textwidth} p{.1\textwidth}} 
Where  & & \\
$u(x)$ is a control signal where safety is ensured  & [$\cdot$] \\
$\tilde{u}(x)$ is a control signal to the linear state space system such that $\tilde{u}=\bar{N}\cdot x_\text{ref}-Kx$ & [$\cdot$] \\ 
$k_0(x)$ is a control law that guarantees safety & [$\cdot$] \\ 
$\sigma(x)$ is a parameter that founds a linear combination of the two controllers & [$\cdot$] 
\end{longtable}
\vspace*{-0.2cm}
The system performs a pure textbook example of a linear state space controller when $\sigma(x) = 0$, thus no safety aspects are considered as the system is within the boundary $\Psi_s$. When $\sigma(x) = 1$ the trajectory has had direct impact with the hard boundary $\Psi_h$ and the system performs with a 100\,\% dedication to $k_0$. This causes the trajectory to oppose its current target and redirect to a safe place in the state space.

\begin{figure}[H]
	\center
		\includegraphics[scale=1]{control_system.pdf}
	\caption{nice figure}
	\label{fig:controlsystem}
\end{figure}

In the safe area 
\begin{figure}[H]
\center
\includegraphics[scale=0.5]{parabel_1.eps}
\caption{nice figure}
\label{fig:lol}
\end{figure}
This allow for a control system as depicted in 
