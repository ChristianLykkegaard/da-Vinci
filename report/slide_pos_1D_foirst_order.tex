\begin{lstlisting}[language=matlab]
close all;
clear; 
clc; 
format long;
hfile = matlab.desktop.editor.getAll;

model = 2; % 0 = both, 1 = first order model, 2 = second order model

%--- parabola coeficients for position constraints ---%
a = 16/9; b = 4/45; c = -2/225; 
%--- elliptic paraboloid  coeficients for position constraints ---%
x10 = 1/40; x20 = 0; a1 = -3/40; b1 = -10; c1 = 1; c2 = -1

if model == 0
    B = 2;
    B_ = 1;
else
    B = 1;
    B_ = 0;
end

vv = 0;
for v = 1:1:B
    if v == 1 && B_ == 1
        model = 1;
        fprintf('Simulating first order model..\n');
    elseif v == 2 && B_ == 1
        model = 2;
        fprintf('Simulating second order model..\n');
    end
    
if model == 2
    s = tf('s'); % prepare Laplace operator
    ts = (28-9)*1/50; % 5 percent settling time
    tr = 0.1; % rise time
    wn = 1.8/tr; % calculate natural frequency
    zeta = -1/(wn*ts)*log(0.02); % calculate the damping ratio
    H = wn^2/(s^2 + 2*zeta*wn*s + wn^2); % calculate transfer function
    num = wn^2; % Specify numarator
    den = [1 2*zeta*wn wn^2]; % specify denominator
    %[A,B,C,D] = tf2ss(num,den); % convert to state space system
    %sys = ss(A,B,C,D); % make a sys
    A = [0 1; -wn^2 -2*zeta*wn];
    B = [0 wn^2]';
    C = [1 0];
    D = 0;
    sys = ss(A,B,C,D)
    x(1,1) = 0 % initial state position
    x(2,1) = 0; % initial state velocity
    K = acker(sys.a,sys.b,[-40 -50]);
elseif model == 1
    tau = 0.110; % time constant
    a_sys = -1/tau; %
    b_sys = 1/tau; % sine wave frequency
    sys = ss(a_sys,b_sys,1,0);
    x(1,1) = 0; % initial state;
    K = acker(a_sys,b_sys,[10*eig(sys.a)]); % control gain   
end

kappa = 1;
Nbar = - inv(sys.c*inv(sys.a-sys.b*K)*sys.b); % ensure unity gain
scrsz = get(groot,'ScreenSize'); % get screen information

%--- Find epsilon ---%
x_epsilon = 0.04; % find epsilon from desired soft limit
epsilon = a*x_epsilon^2 + b*x_epsilon + c; % find epsilon
syms x0
softlims = solve(a*x0^2 + b*x0 + c == epsilon); % find soft limits
epsilon = abs(epsilon); % specify ep silon as a positive number

XREF = [0.02 0.09 -0.14 -0.02 0.045 0.01]; % simulation setpoints
%XREF = [0.025 1 0.0 -0.2 0.045 0.01]; % simulation setpoints
xref = XREF(1); % initial reference

f = 5000; Ts = 1/f; % sampling frequency
N = 1; % simulation time in seconds
fprintf('Simulation time: %d seconds\n', N)

i = (0:Ts:N); % make simulation resolution realistic
utilde = zeros(round(length(i)),1); % init utilde
Rplot(1) = 1; % init reference plot

for R = 1:length(i)
  REFS = 6;
  if R == round(length(i)/REFS)*1
      xref = XREF(2);
      Rplot(2) = R;
  elseif R == round(length(i)/REFS)*2
      xref = XREF(3);
      Rplot(3) = R;
  elseif R == round(length(i)/REFS)*3
      xref = XREF(4);
      Rplot(4) = R;
  elseif R == round(length(i)/REFS)*4
      xref = XREF(5);
      Rplot(5) = R;
  elseif R == round(length(i)/REFS)*5
      xref = XREF(6);
      Rplot(6) = R;
  end
  
  if 1
    if model == 2
      max_vel = 2;
      if x(2,R) > max_vel
          x(2,R) = max_vel;
      elseif x(2,R) < -max_vel
          x(2,R) = -max_vel;
      end
    end
  end
 
  %--- determine sigma  ---%
  y(1,R) = sys.c*x(:,R);
  
  if model == 1
      if (a*(x(1,R))^2 + b*(x(1,R)) + c) <= -epsilon
          sigma = 0;
      elseif ((a*(x(1,R)).^2 + b*(x(1,R)) + c) > -epsilon) && ...
             ((a*(x(1,R)).^2 + b*(x(1,R)) + c) <  0)
          sigma = -2*((a*(x(1,R)).^2 + b*(x(1,R)) + c)/epsilon).^3 - ...
                   3.*((a*(x(1,R)).^2 + b*(x(1,R)) + c)/epsilon ).^2 + 1;
      else
          sigma = 1;
      end
  elseif model == 2
      cbf = (a.*(x(1,R)).^2 + b.*x(1,R) + c);
      if cbf <= -epsilon
          sigma = 0;
      elseif (cbf > -epsilon) && (cbf < 0)
          if model == 1
              sigma = -2*((a*(x(1,R)).^2 + b*(x(1,R)) + c)/epsilon).^3 - ...
                       3.*((a*(x(1,R)).^2 + b*(x(1,R)) + c)/epsilon ).^2 + 1;
          elseif model == 2
              sigma = -2*((a*(x(1,R)).^2 + b*(x(1,R)) + c)/epsilon).^3 - ...
                       3.*((a*(x(1,R)).^2 + b*(x(1,R)) + c)/epsilon ).^2 + 1;
          end
      else
          sigma = 1;
      end 
  end

  if mod(R,1000) == 1 
      if R ~= 1
          fprintf('iter = %d of %d\n', R-1, length(i)-1);
      else
          fprintf('iter = %d of %d\n', R, length(i)-1);
      end
  end

  if model == 2
      LgB(1,R) = (c1*wn^2*(2*x(2,R) + 2*x20))/b1^2;
      LfB(1,R) = (c1*x(2,R)*(2*x(1,R) + 2*x10))/a1^2 - ...
          (c1*(2*x(2,R) + 2*x20)*(x(1,R)*wn^2 + 2*x(2,R)*zeta*wn))/b1^2;  
  elseif model == 1
      LgB(1,R) = (2*(a)*(x(:,R)) + (b))*(sys.b);
      LfB(1,R) = (2*(a)*(x(:,R)) + (b))*((sys.a)*x(:,R));
  end
  
  %-- Find controller by pole placement --%
  utilde(1,R) = xref*Nbar - K*x(:,R);

  %--- Find safe controller ---%
  threshold = 0.0001;
  gamma2 = 1;
  gamma1 = 0.5;
  if abs(LgB(1,R)) >= threshold
      k0(1,R) = -( ( LfB(1,R) + sqrt(LfB(1,R)^2 ...
          + kappa^2*LgB(1,R)*LgB(1,R)' )) /  (LgB(1,R)*LgB(1,R)')  ) *LgB(1,R);
      kplot(1,R) =  0;%k0(1,R);
  else
      k0(1,R) = x(1,R);
      kplot(1,R) = k0(1,R);
  end 
  
  u0(1,R) = sigma*k0(1,R)+(1-sigma)*utilde(1,R);
  
  if 1
    slide_lim = 0.1;
    if u0(1,R) > slide_lim
      u0(1,R) = slide_lim;
    elseif u0(1,R) < -slide_lim
      u0(1,R) = -slide_lim;
    end
  end
  
  LfclB(1,R) = LfB(1,R) + LgB(1,R).*k0(1,R);
  
  %--- Forward Euler ---%
  xdot = sys.a*x(:,R) + u0(1,R)*sys.b;
  x(:,R+1) = xdot*Ts + x(:,R);
  sig(1,R) = sigma;
  
end

%--- plot LfB(x) and LgB(x) ---%
M = 2.5;
figure('Position',[scrsz(1) scrsz(4) scrsz(3)/3 scrsz(4)/3])
hold on
plot(y(1,1:end),LfB(1,:),'-b','linewidth',1)
plot(y(1,1:end),LgB(1,:),'-g','linewidth',1)
x_p = find(LfB >= 0);
plot(y(x_p),LfB(x_p),'-r','linewidth',1)
plot([-0.11 0.11], [0 0], ':k')
plot([-b/(2*a) -b/(2*a)],[-M M],'-.m');
plot([0.04 0.04],[-M M],'-.r');
plot([0.05 0.05],[-M M],'-.r');
plot([-0.09 -0.09],[-M M],'-.r');
plot([-0.1 -0.1],[-M M],'-.r');
h_legend = legend({'$L_fB(x)<0$','$L_gB(x)$','$L_fB(x) \geq 0$','zero level'}...
    ,'Interpreter','latex','location','northeast');
hold off
set(h_legend,'FontSize',14);
ax = gca;
%ax.YLim = [-M M];
ax.XLim = [-0.11 0.11];
fig = gcf;
fig.Name = 'Lie derivatives';
xlabel('slide position [m]')
ylabel('$L_fB(x) \, \wedge \, L_gB(x)$','Interpreter','latex')
set(gca,'fontsize',14)
title('Lie derivatives', 'FontSize', 14);
if 1
    if  model == 1
        str1 = 'CBF extremity','Interpreter','latex';
        text(-b/(2*a),-1,str1);
    end
    str1 = '\Lambda_{s+}','Interpreter','latex';
    text(0.04,-1,str1);
    str1 = '\Lambda_{h+}','Interpreter','latex';
    text(0.05,-1,str1);
    str1 = '\Lambda_{s,}','Interpreter','latex';
    text(-0.09,-1,str1); 
    str1 = '\Lambda_{h-}','Interpreter','latex';
    text(-0.1,-1,str1); 
end
set(findall(gcf,'type','text'),'FontSize',14,'fontWeight','normal')
set(0,'defaultAxesFontName', 'Times New Roman')
set(0,'defaultTextFontName', 'Times New Roman')
box on


%--- plot Barrier function ---%
z = linspace(-1, 1, 10^4);
fprintf('plotting barrier function..\n')
figure('Position',[scrsz(1) scrsz(4)/10 scrsz(3)/3 scrsz(4)/3])
bar = a.*z.^2 + b.*z + c;
h0 = plot(z,bar,'-b', 'linewidth',1);
h0.DisplayName = 'barrier function';
hold on;
h1 = plot([-10 10],[0 0],'--k', 'linewidth', 1);
h1.DisplayName = 'zero line';
slide_lim_y = [-0.2 0.2]; 
slide_lim = [ 0.1  0.1];
h2 = plot(slide_lim, slide_lim_y, '-r', -slide_lim, slide_lim_y, '-r');
h2(1).DisplayName = 'slide limits';
plot(y(1,:),zeros(size(y)), 'xg')
plot(y(1,end),0, 'xk')
plot([x(1,end) x(1,end)], [-10 10], '-.r')
axis([-0.15 0.1 -0.015 0.02]);
grid off;
legend([h0, h1, h2(1)],...
        h0.DisplayName, h1.DisplayName, h2(1).DisplayName,...
       'Location', 'NorthEast');
fig = gcf;
fig.Name = 'Overview';
hold off;

%--- plot control signal ---%
fprintf('plotting control signals..\n')
figure('Position',[scrsz(1)*scrsz(4)/1.5 scrsz(4) scrsz(3)/5 scrsz(4)/2.8])
hold on
if vv == 1
    plot(linspace(0,length(i),length(u01(1,:))), u01(1,:),'-.r');
end
plot(linspace(0,length(i),length(u0(1,:))), u0(1,:),'-b');
fig = gcf;
fig.Name = 'control signal';
if vv == 1
    legend('control signal based on 1st order model','control signal based on 2nd order model')
else
    legend('control signal')
end
h_legend = legend({'$u(x),\,\,\,\,  \kappa = 1$'},'Interpreter','latex');
set(h_legend,'FontSize',14);
ax = gca;
fig = gcf;
%ax.XLim = [0 1000];
fig.Name = 'Control signals';
xlabel('time [s/1000]')
ylabel('u(x)')
set(gca,'fontsize',14)
title('Control signal', 'FontSize', 14);
box on
hold off

%--- plot state trajectory with boundaries ---%
fprintf('plotting states..\n')
figure('Position',[scrsz(1)*scrsz(4)*1.1 scrsz(4)/10 scrsz(3)/3 scrsz(4)/3])
hold on
plot(linspace(0,length(i),length(y(1,:))), y(1,:),'-b','linewidth',1);
plot(linspace(0,length(i),length(sig(1,:))), sig(1,:)./10,'-','Color',...
    [0/255,255/255,0/255],'linewidth',1);
if vv == 1
   plot(linspace(0,length(i),length(y1(1,:))), y1(1,:),'-.b','linewidth',1);
   plot(linspace(0,length(i),length(sig1(1,:))), sig1(1,:)./10,'-.','Color',...
        [0/255,255/255,0/255],'linewidth',1);
end
plot([0 length(i)], [-0.1 -0.1],'-r')
plot([0 length(i)], [softlims(1) softlims(1)],'-.r')
plot([0 length(i)], [-0.1 -0.1],':k')
Rplot(length(XREF) + 1) = length(i);
for j = 1:length(XREF)
    plot([Rplot(j) Rplot(j+1)],[XREF(j) XREF(j)],'-k', 'linewidth',1);
    plot(Rplot(j), XREF(j), '*k');
end
plot([0 length(i)], [softlims(2) softlims(2)],'-.r')
plot([0 length(i)], [0.05 0.05],'-r')
plot([0 length(i)], [0.1 0.1],':k')
h_legend = legend({'slide position','$\sigma(x)\cdot0.1$','$\Lambda_{h\pm}$','$\Lambda_{s\pm}$',...
    'physical limit','reference level'},'Interpreter','latex','location','southeast');
set(h_legend,'FontSize',14);
ax = gca;
ax.YLim = [-0.15 0.11];
ax.XLim = [0 length(i)];
fig = gcf;
fig.Name = 'states';
xlabel('time [s/1000]')
ylabel('slide position [m]')
set(gca,'fontsize',14)
if model == 1
    title('State trajectory for slide position based on first order model', 'FontSize', 14);
elseif model == 2
    title('State trajectory for slide position based on second order model', 'FontSize', 14);
end    
box on
hold off

if model == 2
    figure
    plot(x(2,:))
    title('velocity')
end
figure
plot(LfclB(1,:))
title('LfclB')


figure
hold on
plot(linspace(1,length(i),length(i)),LfB(1,:))
plot(linspace(1,length(i),length(i)),LgB(1,:))
hold off
legend('LfB','LgB')
title('LgB and LfB')


if model == 2
    figure
    plot(linspace(1,length(i),length(i)),((((x(1,1:end-1)+x10).^2))./a1^2 + ((x(2,1:end-1)+x20).^2)./b1^2)*c1 + c2)
    title('B(x1,x2)')
elseif model == 1
    figure
    plot(linspace(1,length(i),length(i)),a.*x(1,1:end-1).^2 + b.*x(1,1:end-1) + c)
    title('B(x1)')
end

end

if v == 1
    y1 = y(1,:);
    sig1 = sig(1,:);
    vv = 1;
    u01 = u0(1,:);
end

fprintf('Done!\n')
set(findall(gcf,'type','text'),'FontSize',14,'fontWeight','normal')
set(0,'defaultAxesFontName', 'Times New Roman')
set(0,'defaultTextFontName', 'Times New Roman')
\end{lstlisting}