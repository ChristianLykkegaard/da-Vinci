\section{Implementation of Safety Controllers}
This appendix contains the MATLAB implementation of the controllers developed throughout the project, i.e.:
\begin{itemize}
\item Safety controller for a system with static boundaries for instrument slide in \autoref{sec:app:static}.
\item Safety controller for a system with dynamic boundaries to simulate a set-up consisting of a beating heart and a safe distance between heart and robot end effector. This is in \autoref{sec:app:dynamic}.
\item Safety controller for a system in 3D Euclidean space comprising control of all joints in the robotic hand, employing forward an inverse kinematics. This is in found \autoref{sec:app:3d}.
\end{itemize}
\subsection{Implementation of Instrument Slide Safety Controller with Static Boundaries}\label{sec:app:static}
This section contains the MATLAB implementation of the slide controller developed in \autoref{chap:cbf_1d_static}. No user inputs are required to run the script. The script shown here includes no plotting, but the script found in \autoref{app:cd} under the path \texttt{matlab\_scripts/slide\_controller/slide\_controller.m} includes all plotting details.\\

\begin{lstlisting}[language=matlab]
model = 2; % 1 = first order model, 2 = second order model

%--- parabola coeficients for position constraints ---%
a = 16/9; b = 4/45; c = -2/225; 
%--- elliptic paraboloid  coeficients for position constraints ---%
x10 = 1/40; x20 = 0; a1 = -3/40; b1 = -10; c1 = 1; c2 = -1;

if model == 2
    s = tf('s'); % prepare Laplace operator
    ts = (28-9)*1/50; % 5 percent settling time
    tr = 0.1; % rise time
    wn = 1.8/tr; % calculate natural frequency
    zeta = -1/(wn*ts)*log(0.02); % calculate the damping ratio
    H = wn^2/(s^2 + 2*zeta*wn*s + wn^2); % calculate transfer function
    num = wn^2; % Specify numarator
    den = [1 2*zeta*wn wn^2]; % specify denominator
    A = [0 1; -wn^2 -2*zeta*wn];
    B = [0 wn^2]';
    C = [1 0];
    D = 0;
    sys = ss(A,B,C,D)
    x(1,1) = 0 % initial state position
    x(2,1) = 0; % initial state velocity
    K = acker(sys.a,sys.b,[-14 -15]);
elseif model == 1
    tau = 0.110; % time constant
    a_sys = -1/tau; %
    b_sys = 1/tau; % sine wave frequency
    sys = ss(a_sys,b_sys,1,0);
    x(1,1) = 0; % initial state;
    K = acker(a_sys,b_sys,[1.1*eig(sys.a)]); % control gain   
end

kappa = 1; % design parameter
Nbar = - inv(sys.c*inv(sys.a-sys.b*K)*sys.b); % ensure unity gain
scrsz = get(groot,'ScreenSize'); % get screen information

%--- Find epsilon ---%
x_epsilon = 0.04; % find epsilon from desired soft limit
epsilon = a*x_epsilon^2 + b*x_epsilon + c; % find epsilon
syms x0
softlims = solve(a*x0^2 + b*x0 + c == epsilon); % find soft limits
epsilon = abs(epsilon); % specify ep silon as a positive number

%--- make reference vector ---%
XREF = [0.02 0.09 -0.14 -0.02 0.045 0.01]; % simulation setpoints
xref = XREF(1); % initial reference

f = 100; Ts = 1/f; % sampling frequency
N = 5; % simulation time in seconds
fprintf('Simulation time: %d seconds\n', N)

i = (0:Ts:N); % make simulation resolution realistic
utilde = zeros(round(length(i)),1); % init utilde
Rplot(1) = 1; % init reference plot

for R = 1:length(i)
  %--- set various references ---%
  REFS = 6;
  if R == round(length(i)/REFS)*1
      xref = XREF(2);
      Rplot(2) = R;
  elseif R == round(length(i)/REFS)*2
      xref = XREF(3);
      Rplot(3) = R;
  elseif R == round(length(i)/REFS)*3
      xref = XREF(4);
      Rplot(4) = R;
  elseif R == round(length(i)/REFS)*4
      xref = XREF(5);
      Rplot(5) = R;
  elseif R == round(length(i)/REFS)*5
      xref = XREF(6);
      Rplot(6) = R;
  end
  
  %--- physical constraints for velocity  ---%
  if 1
    if model == 2
      max_vel = 1;
      if x(2,R) > max_vel
          x(2,R) = max_vel;
      elseif x(2,R) < -max_vel
          x(2,R) = -max_vel;
      end
    end
  end
 
  %--- output ---%
  y(:,R) = sys.C*x(:,R);
  
  %--- determine sigma  ---% 
  if model == 1
      if (a*(x(1,R))^2 + b*(x(1,R)) + c) <= -epsilon
          sigma = 0;
      elseif ((a*(x(1,R)).^2 + b*(x(1,R)) + c) > -epsilon) && ...
             ((a*(x(1,R)).^2 + b*(x(1,R)) + c) <  0)
          sigma = -2*((a*(x(1,R)).^2 + b*(x(1,R)) + c)/epsilon).^3 - ...
                   3.*((a*(x(1,R)).^2 + b*(x(1,R)) + c)/epsilon ).^2 + 1;
      else
          sigma = 1;
      end
  elseif model == 2
      cbf = (a.*(x(1,R)).^2 + b.*x(1,R) + c);
      if cbf <= -epsilon
          sigma = 0;
      elseif (cbf > -epsilon) && (cbf < 0)
          if model == 1
              sigma = -2*((a*(x(1,R)).^2 + b*(x(1,R)) + c)/epsilon).^3 - ...
                       3.*((a*(x(1,R)).^2 + b*(x(1,R)) + c)/epsilon ).^2 + 1;
          elseif model == 2
              sigma = -2*((a*(x(1,R)).^2 + b*(x(1,R)) + c)/epsilon).^3 - ...
                       3.*((a*(x(1,R)).^2 + b*(x(1,R)) + c)/epsilon ).^2 + 1;
          end
      else
          sigma = 1;
      end 
  end

  %--- print every thousand iteration to user ---%
  if mod(R,1000) == 1 
      if R ~= 1
          fprintf('iter = %d of %d\n', R-1, length(i)-1);
      else
          fprintf('iter = %d of %d\n', R, length(i)-1);
      end
  end

  %--- find lie derivatives ---%
  if model == 2
      LgB(1,R) = (c1*wn^2*(2*x(2,R) + 2*x20))/b1^2;
      LfB(1,R) = (c1*x(2,R)*(2*x(1,R) + 2*x10))/a1^2 - ...
          (c1*(2*x(2,R) + 2*x20)*(x(1,R)*wn^2 + 2*x(2,R)*zeta*wn))/b1^2; 
  elseif model == 1
      LgB(1,R) = (2*(a)*(x(:,R)) + (b))*(sys.b);
      LfB(1,R) = (2*(a)*(x(:,R)) + (b))*((sys.a)*x(:,R));
  end
  
  %-- Find controller by pole placement --%
  utilde(1,R) = xref*Nbar - K*x(:,R);

  %--- Find safe controller ---%
  threshold = 0.001;
  if abs(LgB(1,R)) >= threshold
      k0(1,R) = -( ( LfB(1,R) + sqrt(LfB(1,R)^2 ...
          + kappa^2*LgB(1,R)*LgB(1,R)' )) /  (LgB(1,R)*LgB(1,R)')  ) *LgB(1,R);
      kplot(1,R) =  k0(1,R);
  else
      k0(1,R) = 0;
      kplot(1,R) = k0(1,R);
  end 
  
  %--- control law ---%
  u0(1,R) = sigma*k0(1,R)+(1-sigma)*utilde(1,R);
  
  %--- physical constraints for control signal  ---%
  slide_lim = 0.1;
  if u0(1,R) > slide_lim
      u0(1,R) = slide_lim;
  elseif u0(1,R) < -slide_lim
      u0(1,R) = -slide_lim;
  end
 
  %--- save the LfclB ---%
  LfclB(1,R) = LfB(1,R) + LgB(1,R).*k0(1,R);
  
  %--- extrapolate with forward euler ---%
  xdot = sys.a*x(:,R) + u0(1,R)*sys.b;
  x(:,R+1) = xdot*Ts + x(:,R);
  sig(1,R) = sigma;
  
end
\end{lstlisting}
\subsection{Implementation of Instrument Slide Safety Controller with Dynamic Boundaries}\label{sec:app:dynamic}
This section contains MATLAB implementation of the controller developed in \autoref{chap:cbf_1d_dynamic}. All plotting details are omitted in this section, but the controller can also be found in \autoref{app:cd} under the path \texttt{matlab\_scripts/beating\_heart/beating\_heart\_controller.m} which includes the plotting section as well.\\

\begin{lstlisting}[language=matlab]
%--- setup systems ---%
k = 9;
Nbar = 10;
tau = 0.110;
wh = 2*pi/1.1;
A = [-1/tau  0   0  0;
      0      0   wh 0;
      0     -wh  0  0;
      0      0   0  0];
B = [(1/tau) 0 0 0]';
K = [-k Nbar 0 Nbar];

%--- initial conditions ---%
x(1,1) = 0.05; % initial state position
x(2,1) = -0.03; % initial heart position
x(3,1) = 0; % initial heart velocity
x(4,1) = 0.03; % initial distance

kappa = 1;
scrsz = get(groot,'ScreenSize'); % get screen information

f = 2000; Ts = 1/f; % sampling frequency
N = 5; % simulation time in seconds
fprintf('Simulation time: %d seconds\n', N)

i = (0:Ts:N); % make simulation resolution realistic
utilde = zeros(round(length(i)),1); % init utilde
Rplot(1) = 1; % init reference plot

%--- run controller ---%
epsilon = 0.01; 
for R = 1:length(i)
    
  if mod(R,1000) == 1 
      if R ~= 1
          fprintf('iter = %d of %d\n', R-1, length(i)-1);
      else
          fprintf('iter = %d of %d\n', R, length(i)-1);
      end
  end
  
  %--- give an unsafe distance ---%
  if R > f*2 
      x(4,R) = -0.01;
  end
  
  cbf = x(2,R) - x(1,R);
  %--- determine sigma  ---% 
  if cbf <= -epsilon
      sigma = 0;
  elseif ( (cbf > -epsilon) && (cbf <  0)   )
      sigma = -2*(cbf/epsilon).^3 - 3.*(cbf/epsilon).^2 + 1;
  else
      sigma = 1;
  end
  %sigma = 0;

  %--- find lie derivatives ---%
  LgB(1,R) = -1/tau;
  LfB(1,R) = wh*x(3,R) + x(1,R)/tau;
  
  %-- Find controller by pole placement --%
  utilde(1,R) = K*x(:,R);

  %--- Find safe controller ---%
  threshold = 0.001;
  if abs(LgB(1,R)) >= threshold
      k0(1,R) = -( ( LfB(1,R) + sqrt(LfB(1,R)^2 ...
          + kappa^2*LgB(1,R)*LgB(1,R)' )) /  (LgB(1,R)*LgB(1,R)')  ) *LgB(1,R);
      kplot(1,R) =  k0(1,R);
  else
      k0(1,R) = 0;
      kplot(1,R) = k0(1,R);
  end 

  %--- control law ---%
  u0(1,R) = sigma*k0(1,R)+(1-sigma)*utilde(1,R);
  
  %--- save the LfclB ---%  
  LfclB(1,R) = LfB(1,R) + LgB(1,R).*k0(1,R);
  
  %--- extrapolate with forward euler ---%
  xdot = A*x(:,R) + u0(1,R)*B;
  x(:,R+1) = xdot*Ts + x(:,R);
  
  %--- record sigma ---%
  sig(1,R) = sigma;

  %--- calculate distance between heart and robot end effector ---%
  Delta(1,R) = x(1,R) - x(2,R);
end
\end{lstlisting}

\subsection{Implementation of 3D Euclidean Space Safety Controller}\label{sec:app:3d}

This section contains the MATLAB implementation of the controller of the full robotic hand and instrument in 3D Euclidean space developed in \autoref{chap:cbf_3d_static}. All plotting details are omitted in this section, but the controller can also be found in \autoref{app:cd} under the path \texttt{matlab\_scripts/safe\_3d/safety\_in\_3d.m}.
\begin{lstlisting}[language=matlab]
close all;
clear; 
clc; 
format long;
hfile = matlab.desktop.editor.getAll;

%--- setup systems ---%
taux = 0.070;
tauy = 0.100
tauz = 0.041

A = [-1/taux    0         0 ;
0       -1/tauy     0 ;
0        0        -1/tauz];
B = [ 1/taux    0         0 ;
0        1/tauy     0 ;
0        0         1/tauz];
K = place(A,B,[-15 -15 -15])
C = eye(3);
x = [0.05 0.05 0.02]'; % initial conditions
xref = [0.05 0.05 0.02]'; % initial reference

Nbar = - inv(C*inv(A-B*K)*B); % ensure unity gain
kappa = 1;
scrsz = get(groot,'ScreenSize'); % get screen information

f = 2000; Ts = 1/f; % sampling frequency
N = 10; % simulation time in seconds
fprintf('Simulation time: %d seconds\n', N)

i = (0:Ts:N); % make simulation resolution realistic
%utilde = zeros(3,round(length(i))); % init utilde
Rplot(1) = 1; % init reference plot

cx = 0;
cy = 0;
cz = 0;
rx = 0.03;
ry = 0.06;
rz = 0.03;

%--- run controller ---%
for R = 1:length(i)
xref_vec(:,R) = xref;
if mod(R,1000) == 1 
if R ~= 1
fprintf('iter = %d of %d\n', R-1, length(i)-1);
else
fprintf('iter = %d of %d\n', R, length(i)-1);
end
end

if R > f*1 
xref = [0.06 0.02 0.0]';
end
if R > f*1.5 
xref = [-0.055 0.03 0.01]';
end
if R > f*2.5
xref = [-0.01 -0.01 0]';
end
if R > f*3.5
xref = [-0.055 0 -0.01]';
end  
if R > f*4.5
xref = [0 -0.09 0]';
end
if R > f*5.5
xref = [0 0.09 0.0001]';
end
if R > f*6.5
xref = [0.01 0.09 0.01]';
end       

cbf = -(((x(1,R)-cx)/rx)^2 + ((x(2,R)-cy)/ry)^2 + ((x(3,R)-cz)/rz)^2 -1);

%--- determine sigma  ---% 
epsilon = 0.777777777777778;
if cbf <= -epsilon
sigma = 0;
elseif ( (cbf > -epsilon) && (cbf <  0)   )
sigma = -2*(cbf/epsilon).^3 - 3.*(cbf/epsilon).^2 + 1;
else
sigma = 1;
end

LgB(R,:) = [(2/(taux*rx^2))*(cx-x(1,R)) ...
(2/(tauy*ry^2))*(cy-x(2,R)) ...
(2/(tauz*rz^2))*(cz-x(3,R))];
LfB(1,R) = -2*( ((cx*x(1,R)-x(1,R)^2)/(taux*rx^2)) + ...
((cy*x(2,R)-x(2,R)^2)/(tauy*ry^2)) + ...
((cz*x(3,R)-x(3,R)^2)/(tauz*rz^2)));

utilde(:,R) = -K*x(:,R) + Nbar*xref(:);

%--- Find safe controller ---%
threshold = 0.000001;
%abs(LgB(R,:));

if norm(LgB(R,:),2) >= 0.0000001
k0(:,R) = -( ( LfB(1,R) + sqrt(LfB(1,R)^2 ...
+ kappa^2*LgB(R,:)*LgB(R,:)' )) /  (LgB(R,:)*LgB(R,:)')  ) *LgB(R,:)';
else
k0(:,R) = [0 0 0]';
disp(R);
end 
%sigma = 0;
u0(:,R) = sigma*k0(:,R)+(1-sigma)*utilde(:,R);

%--- extrapolate with forward euler ---%
xdot = A*x(:,R) + B*u0(:,R);
x(:,R+1) = xdot*Ts + x(:,R);

sig(1,R) = sigma;
xref_plot(1,R) = xref(1);
xref_plot(2,R) = xref(2);
xref_plot(3,R) = xref(3);
end
\end{lstlisting}