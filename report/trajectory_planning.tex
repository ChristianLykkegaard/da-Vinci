Do we want to do some trajectory planning?

Sequence of tasks for static trajecory tracking
\vspace{-3mm}
\begin{itemize}
	\itemsep-1.3mm
	\item User inputs desired final (3D) setpoint for the end-effector (in inertial frame)
	\item EITHER: The ROS OMPL computes a trajectory avoiding defined obstacles \\
	OR: $\qquad$We design an algorithm that does the path planning, based on a Lyapunov function in combination with a barrier certificate
	\item From the computed trajectory, a sequence of appropriate setpoints for the low-level controllers are generated
	\item ROS uses its inverse kinematics solvers to compute the desired state for each of the generated setpoints
	\item The low-level controllers implemented in the hardware handles the control of the setpoint tracking
\end{itemize}

Sequence of tasks for dynamic surface point tracking
\vspace{-3mm}
\begin{itemize}
	\itemsep-1.3mm
	\item User inputs desired relative position (distance and orientation) of end-effector frame with respect to the heart (surface point) frame, see \autoref{app:kinematic_model_robot} and \ref{app:dynamic_model_heart}
	\item Configuration of heart frame is computed relative to time (determine $t_0$), and from this the absolute end-effector frame is computed (real-time) - also include barrier certificates to avoid heart penetration?
	\item ROS uses its inverse kinematics solvers to compute the desired state for this configuration (real-time)
	\item The low-level controllers implemented in the hardware handles the control of the configuration tracking (real-time)
\end{itemize}

It is desired at the end of the project to perform a trajectory following task with the robot arm (not only simulations or visualization through the ROS tool Rviz), and for that a tissue phantom must be procured, which allows for multiple tests with cutting tools.


\section{Model of the da Vinci surgical robotic arm}
Construct state space model! Explain why it can be simplified to only include kinematics and not dynamics.

If the system always moves very slowly, acceleration may be ignored and maybe dyamics could be ignored altogether, only leaving the evolution of kinematics in the function $f(x)$.