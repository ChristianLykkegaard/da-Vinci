In minimally invasive surgery (MIS), as opposed to traditional open surgery, only small incisions are made in the patient's abdomen or pelvis in order to gain access to the area under surgery, hence causing less trauma beyond this confined area. This in general provides the patient with quicker recovery, shorter hospital stay and less scarring.
One type of MIS is laparoscopy, invented in the beginning of the 20th century \citep{bib:laparoscopy}, where thin metal telescopes (laparoscopes) with specialized surgical tools attached are inserted into the patient through trocars, allowing the surgeon to maneuver the tools in the inflated abdomen guided by visual feedback from a flexible miniature camera (endoscope) inserted alongside the surgical tools \citep{bib:fascrs}.
In the 1980s robotic laparoscopic surgery was introduced as a master-slave system, where the surgeon controls a robot arm holding the surgical tools from a master console, instead of manipulating the instruments manually.

\section{Highlights in the Development of Surgical Robotics}
While the idea of roboticized telemedicine dates back to 1925 \citep{bib:telemed_predict}, the development of telesurgery was founded by NASA (National Aeronautics and Space Administration) in the 1970s \citep{bib:telesurg_history} combining research within virtual reality, robotics and medicine \citep{bib:brown_univ}, and the first robotic surgery procedure was accomplished in 1985 \citep{bib:telesurg_history}, followed by the first laparoscopic robotic surgical procedure in 1987 \citep{bib:brown_univ}.
The major research within telesurgery was funded by DARPA (Defence Advanced Research Project Administration, research administration under the U.S. Department of Defence), and two main teleoperation systems were developed from this research: da Vinci (from Intuitive Surgical) and Zeus (from Computer Motion) \citep{bib:telesurg_history}.




DARPA invested in the research: early 1990 promising tech emerged, primarily SRI was investigating the feasibility of telesurgery with master-slave manipulators with stereo vision feedback, and in mid 1990s they licenced the master manipulator design of Madhani (MIT) along with many other patents [SurgRob]
Researchers at NASA began working on telepresence surgery combining virtual reality, robots and medicine, and in the 1990s SRI (Stanford Research Institute) joined. \citep{bib:brown_univ}
U.S. Army developed MASH (Mobile Advanced Surgical Hospital) \citep{bib:brown_univ}, soldier loaded and teleoperated in a vehicle
The first commercially available surgical robot became available in 1992\citep{bib:telesurg_history}
First robotic system approved for general surgery in 1993 (Computer Motion robot) \citep{bib:telesurg_history}
The first generation of surgical robots was notable for performing image-guided precision tasks but was limited by the need for preoperative planning
early 90s: emerging era of laparoscopy, system designed to assist the surgeon by taking control of the laparoscopic camera and responding to voice commands
Next step: telemanipulation, master control physically separated from the slave unit
In the 1990s Japanese National Institute of Advanced Industrial Science and Technology were among the first to develop MR compatible robots [SurgRob]
DLR involved with medical robotics since the 90s
In 1992 ROBODOC was FDA approved to conduct trials, it performed the world's first robotic joint replacement surgery [SurgRob]
In 1993 Madhani had the idea for the slave manipulator arm after watching an episode of M*A*S*H [SurgRob] and created the Black Falcon during his work at MIT, becoming he prototype of the da Vinci arms.
In 1994 NASA built a pilot telesurgery system RAMS (Robotic Assist MicroSurgery), but project stopped late 1990 due to lack of funding [SurgRob]. Intuitive acquired all info of RAMS.
Early report from 1996 demonstrated ability of surgeon in same city to successfully mentor and manipulate endoscopic camera (but latency limits to few 100 km distance) \citep{bib:telesurg_history}.
In 1997 da Vinci first human trial took place in Belgium in 1997 \citep{bib:intuitive_monopoly}
Computer Motion's Zeus was introduced in 1998 and was the first to perform fully endoscopic robotic surgery and the initial beating-heart totally endoscopic coronary bypass procedure \citep{bib:brown_univ}
1998: teleoperation has been proposed as the next step in MIS \citep{bib:black_falcon}.
Experiments with surgical procedure in space on animal in U.S. space shuttle in 1998 [SurgRob]
SRI first developed Green Telepresence system, the very first to allow remote patient care, licenced by Intuitive, and around 1998 SRI develop the light-weight version M7 [SurgRob]
DARPA and NASA sponsoring research project for space surgical robots: BioRobotics Lab in U. Washington developed 22 kg robot Raven for long distance remote control, SRI started to develop M7 in 1998, 15 kg portable and deployable [SurgRob]
First market-ready version of da Vinci began tests in 1999. [SurgRob]
da Vinci approved by the U.S. FDA (Food and Drug Administration) for general laparoscopic surgery in 2000 (was already approved in Europe \citep{bib:intuitive_monopoly}), The da Vinci became the first robotic surgical system cleared by FDA for general laparoscopic surgery. [MDD]
NASA started NEEMO projects in Aquarius undersea lab (19 m) in 2001 to test effects of latency [SurgRob]
Operation Lindbergh, first transatlantic surgical procedure in 2001, Computer Motion's Zeus/Socrates? with French doctors in New York performing (gall bladder removal) operation on patient in Strasbourg, France IRCAD
First European operation in weightlessness performed in 2003 on animal onboard Zero G plane [SurgRob]
In 2004 the 7th NEEMO was performed with Zeus controlled from Ontario 2500 km away, delay of 0.1-2s, performed with telementor with not trained crew [SurgRob]
DARPA Trauma Pod program 2005.
In 2005 a test was performed with roboticists, surgeons, aerospace engineers, networking experts: placing manipulator and controller 100 m apart, and network linked via a drone, using the Raven \citep{bib:docatadist} (compact). 2 Mbps, 20 ms delay manip. and 200 ms delay video. U.S. Army sponsored demonstration in 2005 where Raven was set up in a tent in the desert [SurgRob]
ISS equipped with 150Mbps connection in 2005 [SurgRob]
In 2006 surgeon performed human operation while on Zero G parabolic flight [SurgRob] 
In 2006 first phase of Trauma Pod was successfully finished: equipping and integrating robotic operating room  with da Vinci and assisting Mitsubishi robot. [SurgRob]
In 2006 the first-ever demonstration of unmanned telesurgery with M7 [SurgRob]
NASA's NEEMO-12 mission in 2006 or 2007, Raven (and M7) in Aquarius underwater habitat in Florida and teleoperated from Seattle (comm delays about 1s, approx delays to ISS) [SurgRob] simulated in zero gravity?
2006: estimated 2800 robots in surgey and therapy worldwide incl da Vinci, the only commercially available FDA approved
NASA had its first zero gravity experiment in 2007 performed with M7 [SurgRob] M7 modified to overcome acceleration and movement from vehicle transport and showed robust performance uring parabolic flight.
M7 performed the world's first automated ultrasound guided tumor biopsy in 2007 [SurgRob]
2008 neuroArm was first used to remove a brain tumor [SurgRob]
After 40 years NASA is returning to the idea of telerobotics in space with Robonaut 2, a humanoid robot currently inhabiting ISS, has been trained on telemedicine (2014), sent to ISS in 2011 [SurgRob]
2014: ROBODOC is the first real orthopaedic robot performing human procedures, surgical planning station [SurgRob]

DARPAs Trauma Pod program first phase goal is to demonstrate prototype of trauma pod with da Vinci + stretcher from MASH and cuustom tool changer nurse robot, second phase DARPA fund research aimed at miniaturizing and integrate systems \citep{bib:docatadist}
military tool for injured soldiers: Trauma Pod program, autonomous and semi-autonomous mobile platforms for war.
U.S. Military funded initiative managed by SRI International (founded as Stanford research Institute by Stanford University in 1946, independent of the university in 1970), DARPA's Trauma Pod Program aims to develop unmanned mobile operating room \citep{bib:docatadist}. Most of SRI's development was achieved with Trauma Pod [SurgRob]



\section{Surgical Robots Anno 2015}
Master-slave system consisting of patient manipulator and master controller, with (stereo) visual feedback from slave to master, in more and more cases also force feedback, manipulator arm
Telesurgery/telemanipulation, telementoring (telepresence)
Early systems required the surgeon to be in the same room as the patient, later tests from remote areas
One day, maybe patient-specific models may be created from advanced imaging, allowing  remotely simulating te procedure prior to operating and determine the best surgical strategy (or train robot!)
New generation: surgical care not only to soldiers but also to remote locations lacking specialized physicians
Surgery remains fundamentally the same for centuries still, relying heavily on the experience of the surgeon and the dexterity of their hands \citep{bib:docatadist},
da Vinci has 2-3 arms with surgical tools and an extra with a stereoscopic video camera probe, surgeon controls the machine from a console in the same room as the patient, a new da Vinci costs \$1.5 million (2006).
da Vinci not engineered to extremes of work in the field: weighs about  kg, designed specifically for MIS, remote control requires significant modification.
Raven: developing surgical robot that can operate in harsh environments, miniaturized, mobile.
Two types: preplan operation on preoperative images, or system where robot is directly teleoperated by slaved robot
Raven has two arms, mechanical interface specificated is available upon request, open-source common hw/sw, sharing research innovations, embrace open standards \citep{bib:raven_ii} using ROS, currently Raven-II system is installed at Harvard,, John Hopkins University, U. Nebraska, UCLA, UCSC, UC Berkeley, U. Washington, newer also: U. Central Florida, U. Western Ontario, Stanford and Montpellier France \citep{bib:raven_ii}.
Raven is open-architecture, cable-driven 7-DOF arms, stereo vision \citep{bib:raven_debride}.
Open-source: creating subtasks \citep{bib:raven_debride}.
Current robotic surgical assistants are primarily controlled by surgeons (master-slave with negligible delays) \citep{bib:raven_debride}.
Multilateral manipulation (with two or more arms), important that the arms avoid collision \citep{bib:raven_debride} (workspaces are rarely disjoint).
Sterilization demands actuators and encoders are outside the body, so actuation inside is achieved using long cable and flexible elements that compound uncertainty \citep{bib:raven_debride}.
Most surgical robots such as da Vinci and Raven have 6 DOF per arm plus a grasp DOF \citep{bib:raven_debride}
Existing surgical robot systems can be categorized into a spectrum based on the modality of interaction with the surgeon, ranging from pure teleoperated or master/slave systems that directly reproduce the motions of the surgeon - to supervisory or shared-control systems where the surgeon hold and is in control of the medical instrument while the robot provides assistance - to purely autonomous systems where the medical mortions are planned off-line when detailed pre-operative plans of the surgical procedure can be laid out and executed autonomously without interoperative modification \citep{bib:raven_debride}.
da Vinci Research Kit platform provided by Intuitive built from mechanical components from first gen daVinci \citep{bib:raven_observ}, open-source electronics and sw develope by WPI and John Hopkins Univ., hw: two arms and surgeon console with stereo viewer, master controllers and foot pedals.
2014: units of Raven are now in use at 12 leading medical rootics research institutes (currently marketed by Applied Dexterity) [SurgRob]
Range of fields of Raven: cooperative teleoperation, haptics, virtual fixtures, supervision, machine learning, autonomy [SurgRob]
Telementoring vs teleoperated robotic procedures: latter slightly better but much slower [SurgRob]
Medical requirements document of upcoming space missions are composed using experience gained from NEEMO [SurgRob]
The setting of virtual boundaries for the robot, tool limitations and speed constraints: may reduce the risk of malpractice [SurgRob]
2012: Raven will be the only robo besides da Vinci that can collect data from multiple instruments and various grouped procedures [SurgRob]
Raven is 3rd gen surgical robotics research platform originally developed with DoD funding to demonstrate lightweight and field deployable robotics, developed at U. Washington, using experiments deisgned to teach  computer model to recognize the motions of expert laparoscopic surgeons. [SurgRob] Record a stream of dynamic and ckinematioc data from surgeries, into a multi-state statistical model using Hidden Markov Models [SurgRob]
U. Nebraska project: Natural Orifice Translumenal Endoscopic Surgery, no external incisions, with miniature robots [SurgRob]
CyberKnife stereotactic radiosurgery system, sold more than 150 units (2009) costs 4 million USD [SurgRob]. Cleared by FDA for tumor treatment in 2001, in 2009 was 70 000 treatments and 180 systems installed, by 2012 over 100 000 procedures performed and 244 sold dispite significantly more expensive than da Vinci (4 million USD). It is one of te very few existing systems that really does automated robot surgery.
2012: i-Snake by Imperial COllege London powered by micromotors [SurgRob]
2013: CardioArm snake-like robot-
2014: Flex is the first robot-assisted flexible endoscopic platform for surgical procedures
FDA approved surgical robots e.g. da Vinci, NeuroMate, CyberKnife, ROBODOC, MAKO Arm, SpineAssist (2011)
MAKO startup 2004, knee surgery and implants.
2010: Amadeus development of Titan Medical, supposedly superior to da Vinci, four arms, will use KUKAs arm, not use cable-drive (patented by Intuitive), with snake-like flexible tools, expect first prototypes in U.S. in 2015 (2012), next generation of Amadeus is SPORT, expected on US market 2015, costs 600 000 USD (2013)
Imaging, robotics and simulation teams.
3D Slicer is free software from Harvard Medical.
2010: surgical tools through natural orifices or incisions, configuring themselves, smart-pill technologies, bio-robotics, nanoActuators and nanoSensors.
EuroSurge: european initiative to identify available competences and potential missing topics, collect and organize all available information on European surgical robotics research in the Robotic SurgePedia database. (2012)
Self-assembling swallowable robots.
cognitive surgical robot, compact size, modularity, sterilizable tools, price, vision-scaling, tremor filtering.
Italian Alf-X: eye tracking, operation is stopped when the surgeon's gaze is not fixed at the operation site. Price will be 2/3 of da Vinci. Instruments are magnetically fixed.
2013: A robotic surgical system for performing minimally invasive medical procedures includes a robot manipulator for robotically assisted handling of a laparoscopic instrument having a manipulator arm, wrist and end-effector.
Asian: demonstrated operation between Japan and Thailand, Chinese-Japanese robot tested in clinical trial between the two countries in 2012, Chinese Micro Hand A, 5 da Vinci all in research labs in Japan, japanese IBIS much the same functionality as da Vinci but much cheaper, Korea has 36 da Vinci systems (2013) and acquired ROBODOC from ISS in 2005/6, Korean copy of da Vincy called Eterne, human trials scheduled for 2013.
French IRCAD (founded 1994) is world's no. 1 MIS traning center.
ViRob from Israeli Microbot: microrobotics controlled by electromagnetic fields: vision in pioneering Micro Invasive Surgery.
Actual hand movements are scaled down to micro-movements \citep{bib:intuitive_monopoly}.
Raven costs around 250 000 USD (2012) [Economist] Allows researchers to experiment and collaborate. Pieter Abbeel and Ken Goldberg at the University of California, Berkeley, will try teaching the robot to operate autonomously by mimicking surgeons. And Dr Rosen himself will concentrate on replicating between man and machine the close working relationship that a team of human surgeons enjoys. As Intuitive Surgical's patents gradually expire, however, the University of Washington is considering the possibility of spinning off the Raven into a start-up company.
One potential challenger is the Raven, which, like the Da Vinci, was also originally developed for the U.S. Army. [MDD]
2009: DLR has already built several generations of lightweight robotic arms for ground and space application, e.g. KineMedic and MIRO. U. Nebraska has developed coin-sized wheeled robot for biopsy, can move teleoperated around the organs [SurgRob]

Superiority:
can be much more precise
tremor filtering of the surgeon's control, perfecting it
existing systems function like robotized laparoscopic instruments, miniaturized and electromechanically enhanced hands. Greater dexterity, accuracy, stability than humans. Tools positioned by high-precision motors can reach spaces surgeon hands often cannot \citep{bib:docatadist}
daV beginning to show superiority to  conventional: laparoscopic radical prostatectomy, less blood loss, fewer complications, shorter hospital stays, but futture studies have yet to show the benefits og robotic surgery considering the costs (2006) \citep{bib:docatadist}
Lab experiments suggest that autonomous robot can achieve robustness comparable to human performance \citep{bib:raven_debride}
MIS requires execution of many subtasks (not all suited to autonomous operation) \citep{bib:raven_debride}.
Enhanced dexterity and precision, faster than surgeon (KidsArm, 2014)
Singaporean MASTER system master-slave 2 arms, Indian surgeons removed gastric cancer tumors, operation time from 8h to 17min (2011)

Robotic surgery development areas/challenges:
- Minimizing the volume of the patient manipulator for better access in the operating room for the nurse, for transportability in remote/war areas and space ships
- Dexterity of the hand and especially the instrument part, flexible instruments
- Latency, image coding/decoding, fieroptic allow high bandwidth data transmission with delays below 500 ms
- in addition to technical changes there are many medical-legal, billing and liability issues
- cost-effectiveness
- access to bandwidth/advaced telecommunication (limitation in extremely remote areas), satellite-based telecommunication may has shown feasibility \citep{bib:telesurg_history}, but don't cover all areas \citep{bib:docatadist}
- regulations and adoption (of surgeons)
- lack of tactile feedback (2013)
- Advanced imaging
- Education of surgeons in the hand-eye-coordination using a surgical robot
- Restricted degrees of motion
- systems that can work under conditions very different from those of pristine operating rooms \citep{bib:docatadist}
- reliable transmission of the surgeon's commands to a system often roaming in far-flung places
- surgeons have remotely commanded surgical robots with real patints, but only in well-equipped hospitals with dedicated wired communication channels
- need to improve scientific rigor of experimental validation in robotics (statistically significant results) \citep{bib:raven_ii}.
- dramatically enhanced dexterity, but integration between computation and surgery is essentially unchanged \citep{bib:raven_ii}
- most surgery today use of computatinal tools (intelligent/networked/references to databases/processing of images) is limited to befre or after the procedure and not supported by the robot system itself \citep{bib:raven_ii}
- it was 2-3 times slower than a human (Raven) \citep{bib:raven_debride}, execution speed could be improved with better state estimation (strict constraints due to safety)
- reproduce the extraordinary perception and control skills of humans \citep{bib:raven_debride}
- primary difficulty for Raven: state estimation (cable driven actuators, indirect sensors imprecise - slack/stretch in cables and flexibility of links/joints) since not feasible to install joint sensors. Stereo vision for pose estimate is highly noisy \citep{bib:raven_debride}
- primary difficulty with teleoperation beyond Earth orbit is communication lag time/latency, for Mars orbit it is 6.5-44 min [SurgRob] Compression and decompression of video stream takes 1.2 s.
- 1ms samplig time gives significant bandwidth demand, 10Mbps enough for teleoperation, but with HD requires 40Mbps [SurgRob]
- better tactile feedback because the surgeon needs to feel the tissue and the difference in its stiffness



\subsection{The da Vinci Surgical System}
Late 1980s: Original prototype for da Vinci System developed at the former Stanford Research Institute (SRI) under contract with the U.S. Army. [MDD]
Intuitive Surgical was founded in 1995 \citep{bib:intuitive_monopoly} and began funded by DARPA with initiative called SRI Green Telepresence Surgery - DARPA wanted technology for battlefield. Intuitive shifted focus to commercial use in hospitals. Intuitive developed robot-assisted minimally invasive surgery.
In 1997 the first prototype of da Vinci was ready for animal tests at Intuitive, and first human trial took place in Belgium in 1997 \citep{bib:intuitive_monopoly},
First market-ready version of da Vinci began tests in 1999. [SurgRob]
da Vinci approved by the U.S. FDA (Food and Drug Administration) for general laparoscopic surgery in 2000 (was already approved in Europe \citep{bib:intuitive_monopoly}), and heart surgery in 2002 \citep{bib:brown_univ}. Shortly before going public in 2000 Intuitive was sued for patent infringement by COmputer Motion \citep{bib:intuitive_monopoly}.
2002: one serious case, person dired after surgery wih da Vinci [SurgRob]
Used with more than 100 procedures (FDA approved for a dozen)
December 2008: 111 da Vinci shipments worldwide, over 300 000 procedures performed
Intuitive Surgical and Computer Motion merged in 2003 \citep{bib:telesurg_history} after a series of patent lawsuits against each other/lenghty patent litigation: work on da Vinci \citep{bib:intuitive_monopoly}
New generation da Vinci S ready in 2005 [SurgRob]
March 2009: 1171 units sold worldwide \citep{bib:intuitive_monopoly}
Over 2200 systems in place and over 360 000 procedures performed in 2011 worldwide \citep{bib:raven_ii}

da Vinci and Intuitive story here
\$25 000 ow level controller boards and capacity of 7 laptops, >1.4 million lines of operating code, da Vinci S improved to avoid collisions, more slender (2009) [SurgRob]

Intuitive's patents expire in 2016 \citep{bib:intuitive_monopoly}
The complexity of the system creates a long learning curve for surgeons, who must be willing to commit their time to learning the fnctionalities.
da Vinci is currently the most widely used robotic surgical system in this emerging industry \citep{bib:intuitive_monopoly}
Intuitive surgical is the first-mover in the robotic-assisted surgical console market \citep{bib:intuitive_monopoly}, and generates revenue from both initial capital sales, and recurring revenue from instrumnt, accessory and service.
(2010) da Vinci generally sells for 0.7-2.3 million USD depending on config + EndoWrist instruments which will expire or wear out + service \citep{bib:intuitive_monopoly}. Monopoly over systems and associated services.
2010: 10-15 years Intuitive's key technology patents begin to expire \citep{bib:intuitive_monopoly}
End of 2007 Intuitive held licences for over 200 U.S. patents and over 90 foreign patents. and owned over 140 U.S patents and over 60 foreign.
Intuitive working on increasing number of young doctors familiar with their system and work with univeristies such that when they finish education they will expect that surgery should be done with robotic technology. \citep{bib:intuitive_monopoly}
Many key patents in Intuitive's massive patent portfolio will be expiring in 2015 and 2016. These include patents acquired from IBM and ComputerMotion and other that stemmed from the initial DARPA program that inevitably spawned Intuitive Surgical. TransEnterix and Titan Medical will be introducing their own robotic surgery platforms that blow the da Vinci System out of the water. The systems are light and mobile, cost significantly less (~\$500,000 for TransEnterix's SurgiBot), and utilize one single incision about the size of a quarter. TransEnterix's system also provides the much needed tactile feedback, which will win over surgeons in a snap. [http://www.thewallstreetfox.com/2014/04/why-intuitive-surgical-is-mother-of-all.html]


\section{Focus of this Project}
Our focus:
safety
make forbidden areas where the robot cannot enter

ROS middleware layer as in Raven.
Prior work: designing planning and control algorithms for autonomous execution of several surgical subtasks (knot tying, suturing), advances in motion planning, control and perception: integrated task and motion planning ofhigh level task planning using state machines, and motion planning for low level planning algorithm  \citep{bib:raven_debride}
Raven-II inverse control process (not primarily to estimate the pose, in which case standard estimation methods like Kalman would be appropriate) is to calculate, give an desired true pose, the input pose to send the control sw to reach the desired true pose (detected pose with vision system assumed to be the true pose), estimate between measurements using updates from forward kinematics \citep{bib:raven_debride}.
da Vinci Research Kit: learning from demonstrations/by observation. Targets considered form convex regions spherical/linear. Patient Side Manipulators manipulates the instruments about a fixed point called the remote center of motion \citep{bib:raven_observ}.
DLR MIRo integrates torque sensing capabilities on the joint level, cosisting of actuation- position sensing- and torque sensing modules, can run in torque and impedance control mode. Virtual springs/potential fields are used to impose constraint forces preventing the robot from  entering predefined areas.



\begin{itemize}
\item What is da Vincy?
\item State of the art? (Raven)
\item operate without having to remove the hearth (Sloth of Rafael)
\end{itemize}
One should certainly take the risk of patient trauma when an automated surgery is conducted into account. This is seen in Therac-25. It is therefore a necessity to formally prove that the procedure is safe as seen in \citep{bib:safety}