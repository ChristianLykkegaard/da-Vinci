Minimally invasive surgery
Laparoscopy, known as minimally invasive
Endoscopy

Master-slave system consisting of patient manipulator and master controller, with (stereo) visual feedback from slave to master, in more and more cases also force feedback, manipulator arm
Telesurgery/telemanipulation, telementoring (telepresence)
Early systems required the surgeon to be in the same room as the patient, later tests from remote areas
One day, maybe patient-specific models may be created from advanced imaging, allowing  remotely simulating te procedure prior to operating and determine the best surgical strategy (or train robot!)


The idea of roboticized telemedicine dates back to 1925 \cite{bib:telemed_predict}, and the first robotic surgical procedure was accomplished in 1985 (non-laparoscopic) \cite{bib:telesurg_history}. % in the form of a CT-guided brain biopsy by PUMA. 
Robots for laparoscopic surgery introduced in medicine in 1987 \cite{bib:brown_univ}
The development of telesurgery arose in the 1970s: for environments such as war or natural catastrophes
The foundation fo telesurgery can be raced back to NASA, major development was funded by DARPA (Defence Advanced Research Projec Administration), a research administration under the U.S. DoD (Department of Defence), military tool for injured soldier.
Two main teleoperator surgical systems were developed from the DARPA research: Integrated Surgical Systems (later Intuitive Surgical Inc \cite{bib:brown_univ}) da Vinci and Computer Motion's Zeus. \cite{bib:telesurg_history}
Researchers at NASA began working on telepresence surgery combining virtual reality, robots and medicine, and in the 1990s SRI (Stanford Research Institute) joined. \cite{bib:brown_univ}
U.S. Army developed MASH (Mobile Advanced Surgical Hospital) \cite{bib:brown_univ}, soldier loaded and teleoperated in a vehicle
The first commercially available surgical robot became available in 1992\cite{bib:telesurg_history}
First robotic system approved for general surgery in 1993 (Computer Motion robot)
The first generation of surgical robots was notable for performing image-guided precision tasks but was limited by the need for preoperative planning
early 90s: emerging era of laparoscopy, system designed to assist the surgeon by taking control of the laparoscopic camera and responding to voice commands
Next step: telemanipulation, master control physically separated from the slave unit
Early report from 1996 demonstrated ability of surgeon in same city to successfully mentor and manipulate endoscopic camera (but latency limits to few 100 km distance)
Computer Motion's Zeus was introduced in 1998 and was the first to perform fully endoscopic robotic surgery and the initial beating-heart totally endoscopic coronary bypass procedure \cite{bib:brown_univ}
da Vinci approved by the U.S. FDA (Food and Drug Administration) for general laparoscopic surgery in 2000, and heart surgery in 2002 \cite{bib:brown_univ}
Operation Lindbergh, first transatlantic surgical procedure in 2001, Computer Motion's Zeus/Socrates? with French doctors in New York performing (gall bladder removal) operation on patient in Strasbourg, France
Intuitive Surgical and Computer Motion merged in 2003 after a series of patent lawsuits against each other: work on da Vinci
In 2006 a test was performed with roboticists, surgeons, aerospace engineers, networking experts: placing manipulator and controller 100 m apart, and network linked via a drone, using the Raven \cite{bib:docatadist} (compact).

DARPAs Trauma Pod program, explain army and space interests, how many surgical robotic systems are out there


Superiority:
less trauma to all else except the spot where operation is intended, less or no scars
can be much more precise
tremor filtering of the surgeon's control, perfecting it

Robotic surgery development areas:
- Minimizing the volume of the patient manipulator for better access in the operating room for the nurse, for transportability in remote/war areas and space ships
- Dexterity of the hand and especially the instrument part, flexible instruments
- Latency, image coding/decoding, fieroptic allow high bandwidth data transmission with delays below 500 ms
- in addition to technical changes there are manymedical-legal, billing and liability issues
- cost-effectiveness
- access to bandwidth/advaced telecommunication (limitation in extremely remote areas), satellite-based telecommunication may has shown feasibility
- regulations and adoption (of surgeons)
- lack of tactile feedback (2013)
- Advanced imaging
- Education of surgeons in the hand-eye-coordination using a surgical robot
- Restricted degrees of motion
- systems that can work under conditions very different from those of pristine operating rooms \cite{bib:docatadist}

Our focus:
safety
make forbidden areas where the robot cannot enter


\begin{itemize}
\item What is da Vincy?
\item State of the art? (Raven)
\item operate without having to remove the hearth (Sloth of Rafael)
\end{itemize}
One should certainly take the risk of patient trauma when an automated surgery is conducted into account. This is seen in Therac-25. It is therefore a necessity to formally prove that the procedure is safe as seen in \citep{bib:safety}