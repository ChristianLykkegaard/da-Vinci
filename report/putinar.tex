The construction of a valid barrier certificate can be a non-trivial task, as it was seen in the construction of a \gls{cbf} for the second order system in \autoref{subsec:cbf-2order}, and with increasing order of the system for which safety need be guaranteed the difficulty rapidly increases. Hence it is desired to use a more methodical approach to constructing barrier certificates, and for this purpose the MATLAB toolbox SOSTOOLS can be used. This toolbox requires that the problem is cast as a \gls{sos} program, which is why this chapter is dedicated to give an introduction to the concept of \gls{sos} and how the barrier certificate definition can be recast as a \gls{sos} problem.

%This chapter describes and establishes the theory applied when the analysis approach is adopted, i.e. to analyse if a given closed loop system is safe.
When a barrier certificate can be found for a closed-loop system $f_{cl}(x)$ according to \autoref{def:barrier_certificate} this signifies a verification that the system, and hence the controller, is safe.

%Barrier certificates can be used to (in)validate the safety compliance of a controller design by testing if a barrier certificate can be found according to \autoref{eq:barrier_constraints} for the closed-loop system $f_{cl}(x)$. 
When the vector field of the closed-loop system is polynomial and the sets $\mathcal{X}$, $\mathcal{X}_0$ and $\mathcal{X}_u$ are described by polynomial (in)equalities, a polynomial barrier certificate can be constructed using \gls{sos} optimization \citep{bib:prajna_framework}. A polynomial $p(x)$ is \gls{sos} if there exist polynomials $f_1,\dots,f_m$ such that \citep{bib:parrilo_sdp}
\begin{equation}
p(x) = \sum_{j=1}^{m}f_j^2(x)
\end{equation}
%
%A \gls{sos} program is a convex optimization problem of the form \citep{bib:prajna_framework,bib:sostools}
%\begin{subequations}
%\begin{align}
%&\min_{c}\, w^Tc\\
%&\text{subject to} \qquad
%q_{i,0}(x) + \sum_{j=1}^{m} q_{i,j}(x)g_j(x) \,\,\,\in \Sigma[x]
%\qquad \text{for}\quad
%i=1,\dots, p
%\end{align}
%\end{subequations}
%\vspace*{-4mm}
%\begin{tabular}{rl}
%where &\\
%$w$ & is a vector of weighting coefficients of the linear objective function\\
%$c$ & is a vector formed by the (unknown) scalar real coefficients of $g_j(x)$\\
%$g_j(x)$ & are polynomials in $x$\\
%$q_{i,j}(x)$ & are given \gls{sos} polynomials with fixed coefficients\\
%$\Sigma$ & denotes the set of \gls{sos} variables\\
%\subsection*{Notation}

A short illustrating example is given initially with the intention to provide some understanding about \gls{sos} polynomials and the notation applied.
\vspace{-0.0cm}
\begin{exa}[Sum of Squares Polynomial]
Consider the two parabola polynomials $f_1(x)=a\,x^2 + b\, x + c$ and $f_2(x) = (d + e\,x)^2= d^2 + 2de\,x + e^2 x^2$, then if the relationship:
\begin{flalign*}
a = e^2 \mm \wedge \mm b =  2de \mm \wedge \mm  c = d^2 \kk \Big( \text{or simply $b=2\sqrt{ac}$} \Big)
\end{flalign*} 
holds, then $a,b,c$ are \gls{sos} variables, denoted by $a,b,c \in \Sigma$, because they are the coefficients in an \gls{sos} polynomial. In that sense, $\Sigma$ denotes the set of \gls{sos} variables. It can additionally be stated that:
\begin{flalign*}
f_1(x) \in \Sigma[x]
\end{flalign*}
where $\Sigma[x]$ denotes a set of polynomials in $x$ with coefficients in $\Sigma$, i.e. $f_1(x)$ is \gls{sos}.
\end{exa}
\vspace{-0.0cm}
Introducing the notion of a monomial vector as a vector $Z$ in $x$ of degree $deg$; e.g. if $x\in\mathbb{R}^2$ and $deg=[0:2]$ each entry has the form $x_1^ax_2^b$ with exponents $a+b=deg=0,...,2$ i.e.
\begin{equation}
Z=[x_1^0x_2^0\quad x_1^1x_2^0\quad x_1^0x_2^1\quad x_1^2x_2^0\quad x_1^1x_2^1\quad x_1^0x_2^2]^T=[1\quad x_1\quad x_2\quad x_1^2\quad x_1x_2\quad  x_2^2]^T
\label{eq:monomial_example}
\end{equation} 
Now, according to \citep{bib:parrilo_sdp} an \gls{sos} polynomial $p\in \Sigma[x]$ can be formulated on a quadratic form comprising a coefficient matrix and a monomial vector
\begin{equation}
p = Z^T Q \, Z, \qquad\qquad p\geq 0 \quad \forall \, x\in\mathbb{R}^n %\setminus \{0\}
\label{eq:sos_polynomial}
\end{equation}
\begin{tabular}{rl}
where &\\
$Z$ & is a monomial vector in $x\in \mathbb{R}^n$\\
$Q$ & is a real positive semidefinite symmetric coefficient matrix\\
\end{tabular}\\

Now a polynomial barrier certificate can be constructed using Putinar's Positivstellensatz.
A Positivstellensatz is a structure theorem of a positive polynomial on some set, and gives an algebraic certificate that a solution exists for a system of real polynomial inequalities \citep{bib:positivstellensatz}. 
%Obtain certificates of positivity on a basic semialgebraic set $\mathbb{K}\subseteq\mathbb{R}^n$. \citep{bib:sos_putinar_laurent}
%A Positivstellensatz defines the regions of a semialgebraic set where a function is positive. 
%non-commutative Positivstellens\"{a}tze characterize things like a polynomial $p$ being positive where another polynomial $q$ is positive
Specifically for Putinar's Positivstellensatz, a compact set $\mathbb{K}$ is defined by the positivity of some polynomials $g_j$, e.g. in 1D Cartesian space $g(x)$ may be a parabola which is positive-valued on the interval $x\in[a,b]$, hence defining the semialgebraic set $\mathbb{K}=\{x\in[a,b]\}$.
Now the positivity of a polynomial $h$ on the set $\mathbb{K}$ can be expressed in terms of a weighted sum of some polynomials $g_j$ with \gls{sos} coefficients \citep[pp 184-186]{bib:sos_putinar_laurent},\citep[pp 28-29]{bib:sos_putinar_lasserre}.\\

 

\begin{thm}[Putinar's Positivstellensatz]\label{def:putinar}
Given the finite family of polynomials $(g_j)_{j=1}^m$ and the subset $Q(g)$ %is called the quadratic module 
generated by the family $(g_j)_{j=1}^m$ \citep[p 29]{bib:sos_putinar_lasserre}
\begin{subequations}\label{eq:putinar}
\begin{align}
\text{polynomials} \qquad & (g_j)_{j=1}^m \in\mathbb{R}[x]\\
\text{set} \qquad & Q(g)=Q(g_1,...,g_m)\equiv\left\{\left.q_0+\sum\limits_{j=1}^{m}q_jg_j\,\,\right| \, (q_j)_{j=0}^m\in\Sigma[x]\right\}\label{eq:putinar_set_sos}
\end{align}
\end{subequations}
Assume that there exists a function $u(x)\in Q(g)$ such that the level set $\{x\in\mathbb{R}^n \,\,|\,\, u(x)\geq 0\}$ is compact \citep[p 29]{bib:sos_putinar_lasserre}.
Given a polynomial $h$ and the compact basic semialgebraic set $\mathbb{K}$  defined by the nonnegativity of the polynomials $g_1,\dots, g_m$  
\begin{subequations}
\begin{align}
\text{polynomial} \qquad & h \in\mathbb{R}[x]\\
\text{set} \qquad & \mathbb{K}\equiv\left\{\left.x\in \mathbb{R}^n\,\, \right| \, (g_j)_{j=1}^m\geq0\right\}\qquad\qquad\qquad\qquad\qquad\quad
\end{align}
\end{subequations}
If the polynomial $h$ is strictly positive on the set $\mathbb{K}$, then $h\in Q(g)$, which means that $h$  can be formulated as
\begin{equation}\label{eq:sos_barrier}
h = q_0+\sum\limits_{j=1}^{m}q_jg_j
\end{equation}
\end{thm}


%\section{Using Sums of Squares to Construct a Barrier Certificate}
%\vspace*{-7mm}



In \autoref{def:putinar} the \gls{sos} variables $q$ are nonnegative per definition and as seen from \autoref{eq:sos_barrier} $h$ is positive on $\mathbb{K}$ as defined by $(g_j)_{j=1}^m$ being positive  in the region $\mathbb{K}$. Outside $\mathbb{K}$ one or more $g_j$s are negative, and hence the sign of $h$ cannot be determined outside $\mathbb{K}$.
Rearranging \autoref{eq:sos_barrier} to
\begin{equation}
q_0 = h - \sum _{j=1}^{m}q_jg_j \label{eq:putinar_sos}
\end{equation} 
however, the right-hand expression will always be nonnegative due to the SOS equality. Using the Matlab toolbox SOSTOOLS (see \autoref{app:sostools} for an introduction to the toolbox syntax), it is possible to solve for the unknown $h$ with a number of inequalities: expression $\in\Sigma[x]$ (corresponding to expression $\geq 0$) on each set $\mathbb{K}$. 

Now defining the semialgebraic sets $\mathcal{X}$, $\mathcal{X}_u$ and $\mathcal{X}_0$, is a matter of defining one or more functions $g_j$ for each set which are positive on the set. E.g. in order to define the region $\mathcal{X}$ construct a polynomial $g$ such that it is positive within the region and its zero level set constitute the desired border of the region. If several polynomials $g_j$ are used to define $\mathcal{X}$, the set is defined by the positive intersection region, i.e. where all of the $g_j$s are positive valued.

When the polynomials $g_j$ have been defined for each of the sets, the polynomial $h$ in \autoref{eq:putinar_sos} is substituted according to \autoref{def:barrier_certificate}, i.e. when defining $\mathcal{X}$ according to \autoref{cer3}, the polynomial $h$ can be written as $-dB/d x \, f_{cl}$; when defining $\mathcal{X}_0$ use $h=-B$ according to \autoref{cer1}; and when defining $\mathcal{X}_u$ use $h=B$ according to \autoref{cer2}.
In summary, referring to the requirements for a barrier certificate in \autoref{def:barrier_certificate} and the \gls{sos} formulation of the polynomial $h$ in \autoref{eq:putinar_sos} based on Putinar's Positivstellensatz, the inequalities defining the barrier certificate $B(x)$ can be set up as
\begin{subequations}\label{eq:barrier_constraints_putinar}
\begin{flalign}
&&	-B(x) &\geq 0 \kk  \forall \hspace{2mm} x \in \mathcal{X}_0 \qquad\quad \Leftarrow& 	-B(x) - \sum _{j=1}^{m}q_jg_j &\,\,\,\in \Sigma[x] &&& \label{cer1_putinar}\\
&&	B(x) \geq\epsilon&> 0 \kk  \forall \hspace{2mm} x \in \mathcal{X}_u \qquad\quad \Leftarrow& 	B(x)-\epsilon - \sum _{j=1}^{m}q_jg_j &\,\,\,\in \Sigma[x] &&&\label{cer2_putinar} \\
&&	-L_{f_{cl}}B(x) &\geq 0 \kk  \forall \hspace{2mm} x \in \mathcal{X} \qquad\quad\,\, \Leftarrow& 	-L_{f_{cl}}B(x) - \sum _{j=1}^{m}q_jg_j &\,\,\,\in \Sigma[x] &&& \label{cer3_putinar}
\end{flalign}
\end{subequations}
Note that the inequality is on positivity in \autoref{cer2} whereas it is on nonnegativity (being \gls{sos}) in \autoref{eq:putinar_set_sos}. By introducing an arbitrarily small $\epsilon>0$ the positivity constraint can be cast as the nonnegativity constraint in the  SOS inequality of \autoref{cer2_putinar}. %This is, however, not considered an issue in the scope of this project, as the position accuracy of the robot is not on the submillimeter level. 

\textcolor{red}{Bemærk, at denne sætning ikke siger hvor høj grad I skal vælge qerne. (husk at tænke på dette)}



 








 
	


\section{Approach for Verification of System Safety}

The following chapters present the safety verification of first- and second order systems in 1D and 3D with static and dynamic boundaries using Putinar's Positvstellensatz in the SOSTOOLS framework. The same systems are used for the analysis as in \autoref{part:cbf}, and to the extent it is possible, also the same (pole placement design) controllers are tested. \textcolor{red}{Correct this when the chapters are written!!!}

