Barrier certificates can be used to (in)validate the safety compliance of a controller design by testing if a barrier certificate can be found according to \autoref{eq:barrier_constraints} for the closed-loop system $f_{cl}(x)$. This can be done using Putinar's Positivstellensatz to define the spaces $\mathcal{X}$, $\mathcal{X}_u$ and $\mathcal{X}_0$ with \gls{sos} polynomials. 

A Positivstellensatz is a structure theorem of a positive (\gls{sos}) polynomial on some set, and gives an algebraic certificate that a solution exists for a system of real polynomial inequalities \citep{bib:positivstellensatz}. 

\textbf{Stengle's Positivstellensatz}: Given 
\begin{align}
\text{polynomials} \qquad & g_1,...,g_m \in \mathbb{R}[x]\\
\text{set} \qquad & g_J\equiv\prod_{j\in J} \text{ for } J\subseteq\{1...m\}, \text{ with } g_0\equiv 1
\end{align}
The set 
\begin{equation}
T(g_1,...,g_m)\equiv\{\sum\limits_{J\subseteq \{1...m\}}^{}u_Jg_J | u_J\in\Sigma\}
\end{equation} 
is called a preordering on $\mathbb{R}[x]$ generated by $g_1,...,g_m$. 

Theorem: for
\begin{align}
\text{set}	 \qquad & \mathbb{K}=\{x\in \mathbb{R}^n | g_i(x)\geq0\}, i=1...m\\
\text{polynomial} \qquad & p\in\mathbb{R}[x]
\end{align}
then
\begin{align}
p>0 \text{ on } \mathbb{K} & \Leftrightarrow pf=1+g \text{ for some } f,g\in T(g_1,...,g_m)\\
p\geq 0 \text{ on } \mathbb{K} & \Leftrightarrow  pf=p^{2k}+g \text{ for some } f,g\in T(g_1,...,g_m) \text{ and } k\in \mathbb{N}\\
p=0 \text{ on } \mathbb{K} & \Leftrightarrow -p^{2k}\in T(g_1,...,g_m) \text{ for some }  k\in \mathbb{N}
\end{align}
\citep{bib:sos_putinar_laurent}

Theorem: let $k$ be a real closed field,
\begin{align}
\text{polynomial} \qquad & f\in k[x]\\
\text{set}	 \qquad & \mathbb{K}=\{x\in k^n | f_j(x)\geq0, j=1...m\}
\end{align}
define 
\begin{equation}
P(f_1,...,f_m)\equiv \{\sum\limits_{J\subseteq\{i...m\}}^{} q_Jf_J: q_J\in \Sigma[x] \}
\end{equation}
where $\Sigma$ denotes a \gls{sos} variable and $\Sigma[x]$ denotes a polynomial \gls{sos} variable. Then
\begin{align}
\text{Nichtnegativstellensatz:} \qquad\qquad\quad &\nonumber\\
f\geq 0 \text{ on } \mathbb{K} \quad & \textup{ iff } \exists \ell \in \mathbb{N} \textup{ and } g,h\in P(f_1,...,f_m) \text{ such that } fg=f^{2\ell}+h\\
\text{Positivstellensatz:} \qquad\qquad\quad &\nonumber\\
f>0  \text{ on } \mathbb{K} \quad & \textup{ iff } \exists g,h\in P(f_1,...,f_m) \text{ such that } fg=1+h\\
\text{Nullstellensatz:} \qquad\qquad\quad &\nonumber\\
f=0 \text{ on } \mathbb{K} \quad & \textup{ iff } \exists \ell \in \mathbb{N} \textup{ and } g\in P(f_1,...,f_m) \text{ such that } f^{2\ell}+g=0
\end{align}

\citep{bib:sos_putainar_lasserre}

\textbf{Schm\"{u}dgen's Positivstellensatz}: Given
\begin{align}
\text{polynomial} \qquad & p\in\mathbb{R}[x]\\
\text{set} \qquad & \mathbb{K}=\{x\in \mathbb{R}^n | g_i(x)\geq0\}, i=1...m
\end{align}
then
\begin{equation}
p>0 \text{ on } \mathbb{K} \Rightarrow p\in T(g_1,...,g_m)
\end{equation}
\citep{bib:sos_putinar_laurent}

Provide simple characterization of polynomial polysive on a compact basic semialgebraic set $\mathbb{K}$. Let
\begin{equation}
(g_j)_{j=1}^m \subset \mathbb{R}[x]
\end{equation}
be such that the semialgebraic set
\begin{equation}
\mathbb{K}\equiv\{x\in\mathbb{R}^n: g_j(x)\geq 0, j=1...m\}
\end{equation}
is compact. If the polynomial $f\in\mathbb{R}[x]$ is strictly positive on $\mathbb{K}$, then
\begin{equation}
f\in P(f_1,...,f_m)
\end{equation}
that is
\begin{equation}
f = \sum\limits_{J\subseteq\{1...m\}}^{}f_Jg_J \quad \textup{for some SOS } f_J\in\Sigma[x], \textup{ with } g_J=\prod\limits_{j\in J}^{}g_j
\end{equation}
\citep{bib:sos_putainar_lasserre}

\textbf{Putinar's Positivstellensatz}: Given
\begin{align}
\text{polynomials} \qquad & g_1,...,g_m\in\mathbb{R}[x]\\
\text{set} \qquad & M(g_1,...,g_m)\equiv\{u_0+\sum\limits_{j=1}^{m}u_jg_j|u_0,u_j\in\Sigma\}
\end{align}
the set $M(g_j)$ is called the quadratic module generated by $g_1,...,g_m$. Condition:
\begin{equation}
\exists f\in M(g_1,...,g_m) \textup{ s.t. } \{x\in \mathbb{R}^n|f(x)\geq 0 \}
\end{equation}
is a compact set, then $\mathbb{K}$ is compact since
\begin{equation}
\mathbb{K}\subseteq\{x|f(x\geq 0 )\} \quad \forall f\in M(g_1,...,g_m)
\end{equation}
The condition holds if the set $\{x\in\mathbb{R}^n|g_j(x)\geq 0 \}$. Reformulations of the condition:
\begin{align}
\exists N\in \mathbb{N} \quad & \textup{for which } N-\sum\limits_{i=1}^{n}x_i^2\in M(g_1,...,g_m)\\
\forall p\in \mathbb{R}^n \exists N\in \mathbb{N} \quad & \textup{for which } N\pm p\in M(g_1,...,g_m)\\
\exists p_1,...,p_s\in\mathbb{R}^n \quad & \textup{s.t. } p_I\in M(g_1,...,g_m) \forall I\subseteq\{1...m\} \textup{ and }\{x\in\mathbb{R}^n|p_1(x)\geq 0 ... p_s(x)\geq 0 \} \textup{ is compact}
\end{align}
\citep{bib:sos_putinar_laurent}

Associated with the finite family $(g_j)\subset\mathbb{R}[x]$ is the subset
\begin{equation}
Q(g)=Q(g_1,...,g_m)\equiv \{q_0+\sum\limits_{j=1}^{m}q_jg_j: (q_j)_{j=0}^m \subset \Sigma[x] \}
\end{equation}
which is a convex cone called the quadratic module generated by the family $(g_j)$.
\begin{align}
\text{set} \qquad & \mathbb{K}\subset\mathbb{R}^n, \mathbb{K}\equiv\{x\in \mathbb{R}^n : g_j(x)\geq 0, j=1...m \} \\
\text{polynomial} \qquad & f\in \mathbb{R}[x], f=\sum\limits_{J\subseteq \{1...m\}}^{}f_Jg_J \textup{ for some SOS } f_J\in\Sigma[x]
\end{align}
If $f\in\mathbb{R}[x]$ is strictly positive on $\mathbb{K}$ then $f\in Q(g)$, i.e.
\begin{equation}
f = f_0 + \sum\limits_{j=1}^{m}f_jg_j
\end{equation}
for some SOS polynomials $f_j\in\Sigma[x], J=0...m$. \citep{bib:sos_putainar_lasserre}.

Obtain certificates of positivity on a basic semialgebraic set $\mathbb{K}\subseteq\mathbb{R}^n$. \citep{bib:sos_putinar_laurent}
A Positivstellensatz defines the regions of a semialgebraic set where a function is positive. 
non-commutative Positivstellens\"{a}tze characterize things like a polynomial $p$ being positive where another polynomial $q$ is positive


\begin{equation}
f = f_0 + \sum\limits_{i=1}^{}g_i f_i
\end{equation}

the functions $g_i$ describe a constraint on the polynomial function $f$ which we desire to solve for, $f_0,f_i$ are sum of squares (SOS) polynomials. Because $f_0$ is SOS then $f_0=z^TQz>0 \forall z$ where z is the monomial vector of independent state variables and Q is a (real symmetric) coefficient matrix. This means that it is known that for $f>0$ we can write

\begin{equation}
f - \sum\limits_{i=1}^{}g_i f_i >0
\end{equation}

and in the case that $f$ is negative

\begin{equation}
-f - \sum\limits_{i=1}^{}g_i f_i >0
\end{equation}


Sum of squares: require that the 