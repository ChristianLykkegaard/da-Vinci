\section{Step Response of Slide Position}
\begin{itemize}
\item Configure the ROS environment as described in \autoref{app:ros}, i.e. make sure all low level controllers are running, that \texttt{roscore}, the \texttt{davinci\_driver} and the \texttt{moveit\_group} interface is running.
\end{itemize}
Now, open two terminals and prepare both by typing:
\begin{itemize}
\item \texttt{ssh <user name>surgery-srv.lab.es.aau.dk} 
\item \texttt{cd <path to root of workspace>}
\item \texttt{source devel/setup.bash}.
\end{itemize}
\subsection*{First Terminal}
Type:

\hspace{1cm} \texttt{rosrun davinci\_moveit\_config MoveGroupInterfaceExecute}

This launches a \gls{ui}. Type \textbf{c} + \textbf{enter} to enter custom joint angle mode. 

SHOW EXAMPLE!!!! ????
\subsection*{Second Terminal}
By subscribing to the \texttt{joint\_state} topic (\texttt{rostopic echo joint\_states}), all information about the current states can be fetched from the sensors. An example of this is shown below.

\lstdefinestyle{DOS}
{
    backgroundcolor=\color{black},
    basicstyle=\scriptsize\color{green}\ttfamily
}
\begin{lstlisting}[style=DOS]
---
header: 
  seq: 4553
  stamp: 
    secs: 1428950592
    nsecs: 666452523
  frame_id: ''
name: ['p4_hand_pitch', 'p4_hand_roll', 'p4_instrument_jaw_left', 'p4_instrument_jaw_right', 'p4_instrument_pitch', 'p4_instrument_roll', 'p4_instrument_slide']
position: [-0.021504180505871773, 0.027300411835312843, 0.0006707065622322261, -0.00013414131535682827, 0.0012072718236595392, -0.0896063968539238, 1.055011398420902e-05]
velocity: [0.0, 0.0, 0.0, 0.0, 0.0, 0.0, 0.0]
effort: [-0.5, -0.5, -0.5, -0.5, -0.5, -0.5, -0.5]
---
\end{lstlisting}

For this test, it is more appropriate to merely publish the slide position, this can be done by:

\hspace{1cm}\texttt{rostopic echo joint\_states/position[6]}

Which gives an output as shown below.

\begin{lstlisting}[style=DOS]
---
8.20564400783e-06
---
8.20564400783e-06
---
8.20564400783e-06
---
\end{lstlisting}
Instead of leaving the output as a terminal output, the information is mapped to a \texttt{.txt} file with a suitable name, e.g:

\hspace{1cm} \texttt{rostopic echo joint\_states/position[6] > taus\_05cm\_1\_speedlimit\_100.txt}

Use the MATLAB script and the recorded measurement data found under the path \autoref{app:cd} under \texttt{matlab\_scripts/slide\_step/plot\_slide\_pos.m}, to plot the recorded slide position along with an estimated first and second order approximation. The step response is seen in \autoref{fig:stepresponseslideapp}.
\begin{figure}[H]
\center
\includegraphics[scale=0.5]{step_slide.eps}
\caption{Step response from 0\,mm to 5\,mm. Plot details and measurements can be found in \autoref{app:cd} as \texttt{matlab\_scripts/slide\_step/plot\_slide\_pos.m}}. 
\label{fig:stepresponseslideapp}
\end{figure}
The approximated second and first order system are found as...????