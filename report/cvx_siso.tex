\section{Matlab Code - SISO}
The following code is the MATLAB code used to setup and run the SISO simulation. The MATLAB scripts together with the simulation setup  may be found in \autoref{chap:cd} through the path \\ \texttt{MatlabScripts/MPC\_implementation\_SISO\_with\_disturbance}.
\subsection*{\texttt{run\_mpc\_implementation.m}}
\lstset{language=matlab,caption={MPC SISO setup code.},label={code_cvx_full_siso}}
\begin{lstlisting}
%% This function loads the parameters and opens op the simulation
clear all
close all
clc
% Load the parameters and model
n = 2; % nr. of states
p = 1; % nr. of inputs
q = 1; % nr. of outputs

% The model without disturbance model.
% Used in the Observer

A_ = [0.5382 0.0284;...
      -0.7568 1.0115 ];
B_ = [-0.0095;...
      -0.0144];
C_ = [97.9870  -58.7902];

% The model with disturbance model.
% The one implemented into the MPC
A = [0.5382 0.0284 -0.0095;...
      -0.7568 1.0115 -0.0144;...
      0 0 1];
B = [-0.0095;...
      -0.0144;...
      0];
  
C = [97.9870  -58.7902 0];

% Find eigenvalues
lambda = eig(A);
% Define faster eigenvalues
lambdaO = 0.1*lambda;
% Design L
L = -place(A',C',lambdaO)'


% Split L op to L1 and L2
L1 = L(1:2)
L2 = L(3,1)

% Define offsets
m10offset=64;
larg_signal_gain=75;
p10offset=m10offset*larg_signal_gain;

% Open op the simulation
mpc_implementation
\end{lstlisting}

\subsection*{\texttt{cvx\_function.m}}
\lstset{language=matlab,caption={MPC SISO function. MPC ONLINE code.},label={code_cvx_full_siso_mpc}}
\begin{lstlisting}
function [u_out, err1] = cvx_function(x_hat_1,x_hat_2,x_hat_3,u_old, m_evap, clock)
% Load the system used in the MPC
A = [0.5382 0.0284 -0.0095;...
      -0.7568 1.0115 -0.0144;...
      0 0 1];
B = [-0.0095;...
      -0.0144;...
      0];
C = [97.9870  -58.7902 0];

n = 3; % nr. of states
p = 1; % nr. of inputs
q = 1; % nr. of outputs

% Setup the reference moving into the MPC horizon by using the clock input
clock_ = clock/6 % Ts = 6
% Start by setting the whole simulation reference to 40 pa
ref = ones(1,6100)*40;
% Zeroize the 3999 sec
ref(1,1:floor(3999/6)) = zeros(1,floor(3999/6));
% Define the pressure reference depending how far the simulation
% has gone by using the clock input
y_ref = ref(1,1+clock_:10+clock_)
 

% CVX init
N = 10; % Horizon

% The output reference. Pressure ref. smallsingal = 4 Pa
%y_ref = 0*ones(q,N);% pressure ref

% input constraints
%m_evap = 29.6; % mass flow. district water Evaporator
m_abs_setpoint = 1.591*m_evap; % This is from the control computer in Sonderborg
umax = 1.6*m_abs_setpoint*ones(p,1); % 1.6 is the correct value
umin = 0.95*m_abs_setpoint*ones(p,1); % 0.95 is the correct value

% slew-rate constraint
uSlewMax = 0.6535*ones(p,1); % data from Absorber measurements
uSlewMin = -0.6535*ones(p,1); % the correct one is 0.6535

% Gather the states to one state vector
x_hat = zeros(n,1);
x_hat = [x_hat_1;x_hat_2;x_hat_3];
 
cvx_begin quiet
        variables x(n,N) u(p,N) % Set the variables
  minimize(sum (sum_square(C*x-y_ref) +  sum_square([u_old(:) u(:,1:end-1)]-u) ) )
     subject to
        % Model dynamic (equality constraints)
        x == A*[x_hat(:) x(:,1:end-1)] + B*u;
        % Actuator Constraints (inequality constraints)
        repmat(umin,1,N) <= u+64 <= repmat(umax,1,N);
        % Slew-rate constraint (inequality constraints)
        repmat(uSlewMin,1,N) <= [u_old(:) u(:,1:end-1)] - u <= repmat(uSlewMax,1,N);
cvx_end

% Send out the output error
err1 = C*x(:,1) - y_ref(:,1);

% save the next control signal calculated by the MPC
u_out = u(1); %save to system

end
\end{lstlisting}