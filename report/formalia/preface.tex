\newgeometry{left=4.5cm,top=-0.5cm,bottom=2cm,textheight=28cm,includefoot,textwidth=16cm}
\chapter*{Preface}
\vspace{-0.5cm}
The level is a 4$^\text{th}$ semester project rated at 30 ECTS-points within the graduate program in Control and Automation at Aalborg University.

The target group is supervisors, students and other interested parties at the School of Information and Communication Technology at the The Faculty of Engineering and Science.


\subsubsection{Reading Guide}
%\vspace{-0.3cm}
%The project is structured in three major parts where different aspects of the project is considered in each: 
%%\vspace{-0.4cm}
%\begin{itemize}
%\item \ref{part:part1} System Analysis
%\item \ref{part:part2} Controller Design
%\item \ref{part:part3} Discussion 
%\end{itemize}
%\vspace{-0.2cm}
The System Analysis will be dealing with the basic understanding of the problem and model development and  contains thereby the fundamental necessities used to state a requirement specification, which in the end results in guidelines and instructions for the controller design. The controller design contains design considerations and solutions to the issues discussed. Finally, the discussion part contains a summary, discussion and reflection regarding the obtained solutions.

A Symbol- and Acronym list, which features all acronyms used in the report, is found in the very beginning of the report. In the very end of the main report a bibliography is listed which likewise contains all references used in the report. Books are indicated with author, title, publisher, year and ISBN. Web pages are indicated with author, title and year.

Appendices are found after the main report and on an attached CD. The appendices include detailed derivations, source code, a digital version of the report and other materials which are not important for the understanding of the objective of the report.

All figures, tables and equations are referred to by the index of the appropriate chapter followed by a number indicating the number of figure, table or equation in the specific chapter. Thereby they have a unique number which is printed along with its caption (equations have no caption though).
\vspace{-0.5cm}
\subsubsection{Acknowledgements}
\vspace{-0.2cm}
It is the wish of the authors to express a special appreciation to..
\restoregeometry % efter denne side bruges de indstillinger der er sat i preamble
