\section{Applied Methods and Definitions}
\subsubsection*{Work}\label{app:work}
Transferred energy that is driven by difference in pressure or force. Work is donated by $W$ and it has the same unit as energy, [J] or [$\text{W}=\text{J/s}$] depending if it is amount or the rate. Work is divided into shaft-work and flow-work. Shaft-work is used to drive e.g. a pump or a compressor and it is a mechanical energy. The term Flow-work is used when energy is transferred by fluid flowing into or out of a system \citep{Dincer2010}.
\subsubsection*{Heat}\label{app:Heta}
The process when energy transferees between some substance and its surrounding driven by heat difference. Heat is also the amount of the transferred energy, $Q$ \citep{bib:Serway2014}.
\subsubsection*{Specific Heat}\label{app:spe_Heta}
Specific Heat, $c$, is a form of energy, it is an energy needed to raise/drop the temperature of a unit mass of a material by a unit temperature. It has the unit kJ/kg$\cdot$ K or kJ/kg$^\circ$C. The specific heat is also referred as constant-pressure specific heat, $c_p$, or constant-volume specific heat, $c_v$, depending if the process takes place at a constant pressure or volume.
\subsubsection*{Internal Energy}\label{app:int_energy}
\subsubsection*{Enthalpy}\label{app:Enthalpy}
\subsubsection*{Entropy}\label{app:Entropy}
\subsubsection*{Exergy}\label{app:Exergy}
\subsubsection*{Vapor and Steam}\label{app:Vapor}
use ph diagram to explain.
\subsubsection*{Mass Balance}\label{app:Mass_B}
\subsubsection*{Heat Balance}\label{app:Heat_B}
\subsubsection*{Energy Balance}\label{app:Energy_B}
\subsubsection*{Momentum Balance}\label{app:Momentum_B}
\subsubsection*{First Law of Thermodynamic}\label{app:flt}
\subsubsection*{Second Law of Thermodynamic}\label{app:slt}