% GLOSSARY

%\newglossaryentry{<label>}{
%  name={<list entry>},  % the term
%  text={<entry as displayed in report text>},  % optional
%  description={<long description of the entry>} % brief description
%  plural={<entry in plural>} % entry in plural, only used if plural is not just -s
%}

% COMMANDS to use in the report to include the entry
%\gls{<label>} 
%\glspl{<label>} % for plural
%\Gls{<label>}   % capitalized
%\Glspl{<label>} % capitalized plural

\newglossaryentry{laparoscopy}
{name=laparoscopy,
 description={type of MIS where telescopic surgical instruments are inserted via trocars through small incisions in the patient's abdomen or pelvis},
 plural={laparoscopies}
}

\newglossaryentry{endoscope}
{name=endoscope,
 description={miniature telescopic camera used for \gls{laparoscopy}},
 plural={endoscopes}
}

\newglossaryentry{analytic_func}
{name={analytic function},
 description={a function that is infinitely differentiable},
 plural={analytic functions}
}

\newglossaryentry{rational_func}
{name={rational function},
 description={a polynomial (fraction) function, where both numerator and denominator are polynomials},
 plural={rational functions}
}

\newglossaryentry{proper_func}
{name={proper function},
 description={a rational function where the degree of the numerator is less than the degree of the denominator},
 plural={proper functions}
}

\newglossaryentry{injective_func}
{name={injective function},
 description={a function that is one-to-one, i.e. no two (different) points in the domain map to the same point in the codomain},
 plural={injective functions}
}

\newglossaryentry{surjective_func}
{name={surjective function},
 description={a function that is onto, i.e. every point in the codomain is a map of one or more points in the domain},
 plural={surjective functions}
}

\newglossaryentry{bijective_func}
{name={bijective function},
 description={a function that is both injective and surjective, i.e. every point in the codomain is the map of exactly one point in the domain},
 plural={bijective functions}
}

\newglossaryentry{lipschitz}
{name={Lipschitz},
 description={a function whose derivative does not exceed some absolute value}
}

\newglossaryentry{compact_space}
{name={compact space},
 description={a space that is both bounded (does not extend to infinity) and closed (the space includes its boundary)},
 plural={compact spaces}
}

\newglossaryentry{hurwitz}
{name={Hurwitz},
 description={the eigenvalues of a (real square) matrix that is Hurwitz all have negative real part}
}

\newglossaryentry{node}
{name=ROS node,
 description={a process that execute various computations within the ROS environment},
 plural={node}
}


\newglossaryentry{daVinci}
{name=da Vinci,
 description={the name of the surgical robot},
 plural={node}
}

\newglossaryentry{catkin}
{name=catkin,
 description={The build system (low-level) macros and infrastructure used by ROS},
}


\newglossaryentry{dimension}
{name={system dimension},
 description={The dimension of a system refers to the dimension of the physical space it operates in, e.g. a 1D system can move along a line, a 2D system can move on a plane and a 3D system can move in 3D space.},
 plural={dimensions}
}

\newglossaryentry{order}
{name={system order},
 description={The order of a system refers to the dimension of the state-space model of the system, e.g. a second order system can be described in terms of two states i.e. if the state is position and velocity the vector field describes the velocity and acceleration of the state.},
 plural={orders}
}

\newglossaryentry{intrinsic}
{name={intrinsic},
	description={Rotations that are intrinsic refer to rotations about the new/rotated axes, e.g. an intrinsic rotation $xz$ is implemented by first rotating a frame $\Psi_a$ about its $x$-axis, obtaining frame $\Psi_b$, then rotating $\Psi_b$ about the $\Psi_b$ $z$-axis to obtain the $xz$ rotated frame $\Psi_c$.}
}

\newglossaryentry{extrinsic}
{name={extrinsic},
	description={Rotations that are extrinsic refer to rotations about the old/fixed axes, e.g. an extrinsic rotation $xz$ is implemented by first rotating a frame $\Psi_a$ about its $x$-axis, obtaining frame $\Psi_b$, then rotating $\Psi_b$ about the $\Psi_a$ $z$-axis to obtain the $xz$ rotated frame $\Psi_c$. This corresponds to an intrinsic rotation $zx$.}
}

%\newglossaryentry{}
%{name={},
% description={},
% plural={}
%}